%% Generated by Sphinx.
\def\sphinxdocclass{report}
\documentclass[letterpaper,10pt,english]{sphinxmanual}
\ifdefined\pdfpxdimen
   \let\sphinxpxdimen\pdfpxdimen\else\newdimen\sphinxpxdimen
\fi \sphinxpxdimen=.75bp\relax

\usepackage[utf8]{inputenc}
\ifdefined\DeclareUnicodeCharacter
 \ifdefined\DeclareUnicodeCharacterAsOptional
  \DeclareUnicodeCharacter{"00A0}{\nobreakspace}
  \DeclareUnicodeCharacter{"2500}{\sphinxunichar{2500}}
  \DeclareUnicodeCharacter{"2502}{\sphinxunichar{2502}}
  \DeclareUnicodeCharacter{"2514}{\sphinxunichar{2514}}
  \DeclareUnicodeCharacter{"251C}{\sphinxunichar{251C}}
  \DeclareUnicodeCharacter{"2572}{\textbackslash}
 \else
  \DeclareUnicodeCharacter{00A0}{\nobreakspace}
  \DeclareUnicodeCharacter{2500}{\sphinxunichar{2500}}
  \DeclareUnicodeCharacter{2502}{\sphinxunichar{2502}}
  \DeclareUnicodeCharacter{2514}{\sphinxunichar{2514}}
  \DeclareUnicodeCharacter{251C}{\sphinxunichar{251C}}
  \DeclareUnicodeCharacter{2572}{\textbackslash}
 \fi
\fi
\usepackage{cmap}
\usepackage[T1]{fontenc}
\usepackage{amsmath,amssymb,amstext}
\usepackage{babel}
\usepackage{times}
\usepackage[Bjarne]{fncychap}
\usepackage[dontkeepoldnames]{sphinx}

\usepackage{geometry}

% Include hyperref last.
\usepackage{hyperref}
% Fix anchor placement for figures with captions.
\usepackage{hypcap}% it must be loaded after hyperref.
% Set up styles of URL: it should be placed after hyperref.
\urlstyle{same}

\addto\captionsenglish{\renewcommand{\figurename}{Fig.}}
\addto\captionsenglish{\renewcommand{\tablename}{Table}}
\addto\captionsenglish{\renewcommand{\literalblockname}{Listing}}

\addto\captionsenglish{\renewcommand{\literalblockcontinuedname}{continued from previous page}}
\addto\captionsenglish{\renewcommand{\literalblockcontinuesname}{continues on next page}}

\addto\extrasenglish{\def\pageautorefname{page}}

\setcounter{tocdepth}{2}


% Jupyter Notebook prompt colors
\definecolor{nbsphinxin}{HTML}{303F9F}
\definecolor{nbsphinxout}{HTML}{D84315}
% ANSI colors for output streams and traceback highlighting
\definecolor{ansi-black}{HTML}{3E424D}
\definecolor{ansi-black-intense}{HTML}{282C36}
\definecolor{ansi-red}{HTML}{E75C58}
\definecolor{ansi-red-intense}{HTML}{B22B31}
\definecolor{ansi-green}{HTML}{00A250}
\definecolor{ansi-green-intense}{HTML}{007427}
\definecolor{ansi-yellow}{HTML}{DDB62B}
\definecolor{ansi-yellow-intense}{HTML}{B27D12}
\definecolor{ansi-blue}{HTML}{208FFB}
\definecolor{ansi-blue-intense}{HTML}{0065CA}
\definecolor{ansi-magenta}{HTML}{D160C4}
\definecolor{ansi-magenta-intense}{HTML}{A03196}
\definecolor{ansi-cyan}{HTML}{60C6C8}
\definecolor{ansi-cyan-intense}{HTML}{258F8F}
\definecolor{ansi-white}{HTML}{C5C1B4}
\definecolor{ansi-white-intense}{HTML}{A1A6B2}



\title{sharp-repo Documentation}
\date{Sep 14, 2017}
\release{1.0}
\author{Raminder}
\newcommand{\sphinxlogo}{\vbox{}}
\renewcommand{\releasename}{Release}
\makeindex

\begin{document}

\maketitle
\sphinxtableofcontents
\phantomsection\label{\detokenize{index::doc}}


Adaptive reasoning and problem solving represent crucial skills in increasingly information-rich working environments. The SHARP research project is a multi-institutional competitive and collaborative effort that seeks to understand the neurobiological substrates of fluid intelligence and its malleability in response to a wide variety of interventions.  This research effort, across all involved institutions collected the largest dataset on the training of fluid intelligence and its possible neural correlates.  To ultimately share this resources to all researchers, a repository has been created to hold this data: \sphinxurl{https://sharp.bidmc.harvard.edu}.  We have hosted this data using the XNAT data repository infrastructure.  This infrastructure has been extended to fit the data elements of interest, including EEG, quantitative phenotypical information, cognitive assessment and MR imaging data.  At present data includes over 500 research participants, and over 800 imaging sessions. Additional data access tools have been developed for easy researcher which are distributed alongside the data. This repository offers a valuable resource for studies investigating the flexibility of fluid intelligence.

\noindent\sphinxincludegraphics{{xnat_latest}.png}


\chapter{Working with XNAT using PyXNAT}
\label{\detokenize{Sharp_Working_Guide:working-with-xnat-using-pyxnat}}\label{\detokenize{Sharp_Working_Guide:welcome-to-strengthening-human-adaptive-reasoning-and-problem-solving-sharp-repository-documentation}}\label{\detokenize{Sharp_Working_Guide::doc}}

\section{Download PyXNAT Module}
\label{\detokenize{Sharp_Working_Guide:download-pyxnat-module}}
\begin{sphinxVerbatim}[commandchars=\\\{\}]
\PYG{o}{!}pip install pyxnat
\end{sphinxVerbatim}


\subsection{Connect to server: Enter user : password}
\label{\detokenize{Sharp_Working_Guide:connect-to-server-enter-user-password}}
\begin{sphinxVerbatim}[commandchars=\\\{\}]
\PYG{k+kn}{import} \PYG{n+nn}{pyxnat}
\PYG{k+kn}{import} \PYG{n+nn}{os}

\PYG{c+c1}{\PYGZsh{} connect to XNAT instance}
\PYG{k+kn}{from} \PYG{n+nn}{pyxnat} \PYG{k+kn}{import} \PYG{n}{Interface}
\PYG{n}{xnat} \PYG{o}{=} \PYG{n}{Interface}\PYG{p}{(}\PYG{n}{server}\PYG{o}{=}\PYG{l+s+s1}{\PYGZsq{}}\PYG{l+s+s1}{http://sharp.bidmc.harvard.edu:8080}\PYG{l+s+s1}{\PYGZsq{}}\PYG{p}{,}  \PYG{n}{cachedir}\PYG{o}{=}\PYG{l+s+s1}{\PYGZsq{}}\PYG{l+s+s1}{/tmp}\PYG{l+s+s1}{\PYGZsq{}}\PYG{p}{)}
\PYG{n}{xnat}\PYG{o}{.}\PYG{n}{select}\PYG{o}{.}\PYG{n}{projects}\PYG{p}{(}\PYG{p}{)}\PYG{o}{.}\PYG{n}{get}\PYG{p}{(}\PYG{p}{)}
\end{sphinxVerbatim}


\section{Different types of datatypes supported by XNAT}
\label{\detokenize{Sharp_Working_Guide:different-types-of-datatypes-supported-by-xnat}}
\begin{sphinxVerbatim}[commandchars=\\\{\}]
\PYG{n}{xnat}\PYG{o}{.}\PYG{n}{inspect}\PYG{o}{.}\PYG{n}{datatypes}\PYG{p}{(}\PYG{p}{)}
\end{sphinxVerbatim}


\section{Check number of Subject}
\label{\detokenize{Sharp_Working_Guide:check-number-of-subject}}
\begin{sphinxVerbatim}[commandchars=\\\{\}]
\PYG{n}{subjects} \PYG{o}{=} \PYG{n}{xnat}\PYG{o}{.}\PYG{n}{select}\PYG{p}{(}\PYG{l+s+s1}{\PYGZsq{}}\PYG{l+s+s1}{/projects/FAST/subjects}\PYG{l+s+s1}{\PYGZsq{}}\PYG{p}{)}
\PYG{n}{subjects}\PYG{o}{.}\PYG{n}{get}\PYG{p}{(}\PYG{p}{)}\PYG{o}{.}\PYG{n+nf+fm}{\PYGZus{}\PYGZus{}len\PYGZus{}\PYGZus{}}\PYG{p}{(}\PYG{p}{)}
\end{sphinxVerbatim}


\section{Loading the project}
\label{\detokenize{Sharp_Working_Guide:loading-the-project}}
\begin{sphinxVerbatim}[commandchars=\\\{\}]
\PYG{n}{project} \PYG{o}{=} \PYG{n}{xnat}\PYG{o}{.}\PYG{n}{select}\PYG{o}{.}\PYG{n}{project}\PYG{p}{(}\PYG{l+s+s1}{\PYGZsq{}}\PYG{l+s+s1}{FAST}\PYG{l+s+s1}{\PYGZsq{}}\PYG{p}{)}
\PYG{k}{print}\PYG{p}{(}\PYG{n}{project}\PYG{p}{)}
\end{sphinxVerbatim}


\section{Working with Subject Data}
\label{\detokenize{Sharp_Working_Guide:working-with-subject-data}}
\begin{sphinxVerbatim}[commandchars=\\\{\}]
\PYG{n}{contraints} \PYG{o}{=} \PYG{p}{[}\PYG{p}{(}\PYG{l+s+s1}{\PYGZsq{}}\PYG{l+s+s1}{xnat:subjectData/SUBJECT\PYGZus{}ID}\PYG{l+s+s1}{\PYGZsq{}}\PYG{p}{,}\PYG{l+s+s1}{\PYGZsq{}}\PYG{l+s+s1}{LIKE}\PYG{l+s+s1}{\PYGZsq{}}\PYG{p}{,}\PYG{l+s+s1}{\PYGZsq{}}\PYG{l+s+s1}{\PYGZpc{}}\PYG{l+s+s1}{\PYGZsq{}}\PYG{p}{)}\PYG{p}{,}
                  \PYG{l+s+s1}{\PYGZsq{}}\PYG{l+s+s1}{AND}\PYG{l+s+s1}{\PYGZsq{}}\PYG{p}{,} \PYG{p}{(}\PYG{l+s+s1}{\PYGZsq{}}\PYG{l+s+s1}{xnat:subjectData/PROJECT}\PYG{l+s+s1}{\PYGZsq{}}\PYG{p}{,} \PYG{l+s+s1}{\PYGZsq{}}\PYG{l+s+s1}{=}\PYG{l+s+s1}{\PYGZsq{}}\PYG{p}{,} \PYG{l+s+s1}{\PYGZsq{}}\PYG{l+s+s1}{FAST}\PYG{l+s+s1}{\PYGZsq{}}\PYG{p}{)}
              \PYG{p}{]}
\PYG{n}{table} \PYG{o}{=} \PYG{n}{xnat}\PYG{o}{.}\PYG{n}{select}\PYG{p}{(}\PYG{l+s+s1}{\PYGZsq{}}\PYG{l+s+s1}{xnat:subjectData}\PYG{l+s+s1}{\PYGZsq{}}\PYG{p}{,} \PYG{p}{[}\PYG{l+s+s1}{\PYGZsq{}}\PYG{l+s+s1}{xnat:subjectData/SUBJECT\PYGZus{}LABEL}\PYG{l+s+s1}{\PYGZsq{}}\PYG{p}{,}\PYG{l+s+s1}{\PYGZsq{}}\PYG{l+s+s1}{xnat:subjectData/PROJECT}\PYG{l+s+s1}{\PYGZsq{}}\PYG{p}{,}\PYG{l+s+s1}{\PYGZsq{}}\PYG{l+s+s1}{xnat:subjectData/SUBJECT\PYGZus{}ID}\PYG{l+s+s1}{\PYGZsq{}}\PYG{p}{]}\PYG{p}{)}\PYG{o}{.}\PYG{n}{where}\PYG{p}{(}\PYG{n}{contraints}\PYG{p}{)}
\PYG{n}{table}\PYG{o}{.}\PYG{n+nf+fm}{\PYGZus{}\PYGZus{}len\PYGZus{}\PYGZus{}}\PYG{p}{(}\PYG{p}{)}
\end{sphinxVerbatim}

\begin{sphinxVerbatim}[commandchars=\\\{\}]
\PYG{k}{print}\PYG{p}{(}\PYG{n}{table}\PYG{p}{)}
\end{sphinxVerbatim}


\section{Working with MRI Session Data}
\label{\detokenize{Sharp_Working_Guide:working-with-mri-session-data}}
\begin{sphinxVerbatim}[commandchars=\\\{\}]
\PYG{n}{contraints} \PYG{o}{=} \PYG{p}{[}\PYG{p}{(}\PYG{l+s+s1}{\PYGZsq{}}\PYG{l+s+s1}{xnat:mrSessionData/ID}\PYG{l+s+s1}{\PYGZsq{}}\PYG{p}{,}\PYG{l+s+s1}{\PYGZsq{}}\PYG{l+s+s1}{LIKE}\PYG{l+s+s1}{\PYGZsq{}}\PYG{p}{,}\PYG{l+s+s1}{\PYGZsq{}}\PYG{l+s+s1}{\PYGZpc{}}\PYG{l+s+s1}{\PYGZsq{}}\PYG{p}{)}\PYG{p}{,}
                  \PYG{l+s+s1}{\PYGZsq{}}\PYG{l+s+s1}{AND}\PYG{l+s+s1}{\PYGZsq{}}\PYG{p}{,} \PYG{p}{(}\PYG{l+s+s1}{\PYGZsq{}}\PYG{l+s+s1}{xnat:subjectData/PROJECT}\PYG{l+s+s1}{\PYGZsq{}}\PYG{p}{,} \PYG{l+s+s1}{\PYGZsq{}}\PYG{l+s+s1}{=}\PYG{l+s+s1}{\PYGZsq{}}\PYG{p}{,} \PYG{l+s+s1}{\PYGZsq{}}\PYG{l+s+s1}{FAST}\PYG{l+s+s1}{\PYGZsq{}}\PYG{p}{)}
             \PYG{p}{]}
\PYG{n}{table1} \PYG{o}{=} \PYG{n}{xnat}\PYG{o}{.}\PYG{n}{select}\PYG{p}{(}\PYG{l+s+s1}{\PYGZsq{}}\PYG{l+s+s1}{xnat:mrSessionData}\PYG{l+s+s1}{\PYGZsq{}}\PYG{p}{,} \PYG{p}{[}\PYG{l+s+s1}{\PYGZsq{}}\PYG{l+s+s1}{xnat:mrSessionData/SUBJECT\PYGZus{}LABEL}\PYG{l+s+s1}{\PYGZsq{}}\PYG{p}{,}\PYG{l+s+s1}{\PYGZsq{}}\PYG{l+s+s1}{xnat:mrSessionData/SESSION\PYGZus{}ID}\PYG{l+s+s1}{\PYGZsq{}}\PYG{p}{]}\PYG{p}{)}\PYG{o}{.}\PYG{n}{where}\PYG{p}{(}\PYG{n}{contraints}\PYG{p}{)}
\PYG{n}{table1}\PYG{o}{.}\PYG{n+nf+fm}{\PYGZus{}\PYGZus{}len\PYGZus{}\PYGZus{}}\PYG{p}{(}\PYG{p}{)}
\end{sphinxVerbatim}

\begin{sphinxVerbatim}[commandchars=\\\{\}]
\PYG{k}{print}\PYG{p}{(}\PYG{n}{table1}\PYG{p}{)}
\end{sphinxVerbatim}


\section{Filtering using Behavioral scores}
\label{\detokenize{Sharp_Working_Guide:filtering-using-behavioral-scores}}
\begin{sphinxVerbatim}[commandchars=\\\{\}]
\PYG{n}{contraints} \PYG{o}{=} \PYG{p}{[}\PYG{p}{(}\PYG{l+s+s1}{\PYGZsq{}}\PYG{l+s+s1}{xnat:subjectData/SUBJECT\PYGZus{}ID}\PYG{l+s+s1}{\PYGZsq{}}\PYG{p}{,}\PYG{l+s+s1}{\PYGZsq{}}\PYG{l+s+s1}{LIKE}\PYG{l+s+s1}{\PYGZsq{}}\PYG{p}{,}\PYG{l+s+s1}{\PYGZsq{}}\PYG{l+s+s1}{\PYGZpc{}}\PYG{l+s+s1}{\PYGZsq{}}\PYG{p}{)}\PYG{p}{,}
                  \PYG{p}{(}\PYG{l+s+s1}{\PYGZsq{}}\PYG{l+s+s1}{behavioral:scores/VocabScore}\PYG{l+s+s1}{\PYGZsq{}}\PYG{p}{,} \PYG{l+s+s1}{\PYGZsq{}}\PYG{l+s+s1}{\PYGZgt{}=}\PYG{l+s+s1}{\PYGZsq{}}\PYG{p}{,} \PYG{l+s+s1}{\PYGZsq{}}\PYG{l+s+s1}{36}\PYG{l+s+s1}{\PYGZsq{}}\PYG{p}{)}\PYG{p}{,}
                  \PYG{l+s+s1}{\PYGZsq{}}\PYG{l+s+s1}{AND}\PYG{l+s+s1}{\PYGZsq{}}\PYG{p}{,} \PYG{p}{(}\PYG{l+s+s1}{\PYGZsq{}}\PYG{l+s+s1}{xnat:subjectData/PROJECT}\PYG{l+s+s1}{\PYGZsq{}}\PYG{p}{,} \PYG{l+s+s1}{\PYGZsq{}}\PYG{l+s+s1}{=}\PYG{l+s+s1}{\PYGZsq{}}\PYG{p}{,} \PYG{l+s+s1}{\PYGZsq{}}\PYG{l+s+s1}{FAST}\PYG{l+s+s1}{\PYGZsq{}}\PYG{p}{)}
             \PYG{p}{]}
\PYG{n}{table1} \PYG{o}{=} \PYG{n}{xnat}\PYG{o}{.}\PYG{n}{select}\PYG{p}{(}\PYG{l+s+s1}{\PYGZsq{}}\PYG{l+s+s1}{xnat:subjectData}\PYG{l+s+s1}{\PYGZsq{}}\PYG{p}{)}\PYG{o}{.}\PYG{n}{where}\PYG{p}{(}\PYG{n}{contraints}\PYG{p}{)}
\PYG{n}{table1}\PYG{o}{.}\PYG{n+nf+fm}{\PYGZus{}\PYGZus{}len\PYGZus{}\PYGZus{}}\PYG{p}{(}\PYG{p}{)}
\end{sphinxVerbatim}


\section{Downloading the selective data}
\label{\detokenize{Sharp_Working_Guide:downloading-the-selective-data}}
Lets start with download data for one subject

\begin{sphinxVerbatim}[commandchars=\\\{\}]
\PYG{n}{subject} \PYG{o}{=} \PYG{n}{xnat}\PYG{o}{.}\PYG{n}{select}\PYG{o}{.}\PYG{n}{project}\PYG{p}{(}\PYG{l+s+s1}{\PYGZsq{}}\PYG{l+s+s1}{FAST}\PYG{l+s+s1}{\PYGZsq{}}\PYG{p}{)}\PYG{o}{.}\PYG{n}{subject}\PYG{p}{(}\PYG{l+s+s1}{\PYGZsq{}}\PYG{l+s+s1}{0001}\PYG{l+s+s1}{\PYGZsq{}}\PYG{p}{)}
\PYG{n}{experiment} \PYG{o}{=} \PYG{n}{subject}\PYG{o}{.}\PYG{n}{experiment}\PYG{p}{(}\PYG{l+s+s2}{\PYGZdq{}}\PYG{l+s+s2}{SHARP\PYGZus{}E00746}\PYG{l+s+s2}{\PYGZdq{}}\PYG{p}{)}
\PYG{n}{allscans} \PYG{o}{=} \PYG{n}{experiment}\PYG{o}{.}\PYG{n}{scans}\PYG{p}{(}\PYG{p}{)}
\PYG{n}{allscans}\PYG{o}{.}\PYG{n}{download}\PYG{p}{(}\PYG{l+s+s2}{\PYGZdq{}}\PYG{l+s+s2}{/tmp}\PYG{l+s+s2}{\PYGZdq{}}\PYG{p}{,} \PYG{n+nb}{type}\PYG{o}{=}\PYG{l+s+s1}{\PYGZsq{}}\PYG{l+s+s1}{ALL}\PYG{l+s+s1}{\PYGZsq{}}\PYG{p}{,} \PYG{n}{extract}\PYG{o}{=}\PYG{n+nb+bp}{False}\PYG{p}{)}
\end{sphinxVerbatim}

Now lets write a filer to download the selective data for all the
subjects

\begin{sphinxVerbatim}[commandchars=\\\{\}]
\PYG{c+c1}{\PYGZsh{} Filer can be developed based on the data parameters}
\PYG{n}{contraints} \PYG{o}{=} \PYG{p}{[}\PYG{p}{(}\PYG{l+s+s1}{\PYGZsq{}}\PYG{l+s+s1}{xnat:mrSessionData/ID}\PYG{l+s+s1}{\PYGZsq{}}\PYG{p}{,}\PYG{l+s+s1}{\PYGZsq{}}\PYG{l+s+s1}{LIKE}\PYG{l+s+s1}{\PYGZsq{}}\PYG{p}{,}\PYG{l+s+s1}{\PYGZsq{}}\PYG{l+s+s1}{\PYGZpc{}}\PYG{l+s+s1}{\PYGZsq{}}\PYG{p}{)}\PYG{p}{,}
                  \PYG{l+s+s1}{\PYGZsq{}}\PYG{l+s+s1}{AND}\PYG{l+s+s1}{\PYGZsq{}}\PYG{p}{,} \PYG{p}{(}\PYG{l+s+s1}{\PYGZsq{}}\PYG{l+s+s1}{xnat:subjectData/PROJECT}\PYG{l+s+s1}{\PYGZsq{}}\PYG{p}{,} \PYG{l+s+s1}{\PYGZsq{}}\PYG{l+s+s1}{=}\PYG{l+s+s1}{\PYGZsq{}}\PYG{p}{,} \PYG{l+s+s1}{\PYGZsq{}}\PYG{l+s+s1}{FAST}\PYG{l+s+s1}{\PYGZsq{}}\PYG{p}{)}

\PYG{n}{list\PYGZus{}subjects} \PYG{o}{=} \PYG{n}{xnat}\PYG{o}{.}\PYG{n}{select}\PYG{o}{.}\PYG{n}{project}\PYG{p}{(}\PYG{l+s+s1}{\PYGZsq{}}\PYG{l+s+s1}{FAST}\PYG{l+s+s1}{\PYGZsq{}}\PYG{p}{)}\PYG{o}{.}\PYG{n}{subjects}\PYG{p}{(}\PYG{p}{)}\PYG{o}{.}\PYG{n}{where}\PYG{p}{(}\PYG{n}{contraints}\PYG{p}{)}
\PYG{k}{for} \PYG{n}{list\PYGZus{}subject} \PYG{o+ow}{in} \PYG{n}{list\PYGZus{}subjects}\PYG{p}{:}
    \PYG{n}{list\PYGZus{}experiments} \PYG{o}{=} \PYG{n}{list\PYGZus{}subject}\PYG{o}{.}\PYG{n}{experiments}\PYG{p}{(}\PYG{p}{)}\PYG{o}{.}\PYG{n}{where}\PYG{p}{(}\PYG{n}{contraints}\PYG{p}{)}
    \PYG{k}{for} \PYG{n}{list\PYGZus{}experiment} \PYG{o+ow}{in} \PYG{n}{list\PYGZus{}experiments}\PYG{p}{:}
        \PYG{k}{print} \PYG{n}{list\PYGZus{}experiment}
        \PYG{n}{scans} \PYG{o}{=} \PYG{n}{list\PYGZus{}experiment}\PYG{o}{.}\PYG{n}{scans}\PYG{p}{(}\PYG{p}{)}
        \PYG{k}{try}\PYG{p}{:}
            \PYG{c+c1}{\PYGZsh{} Number 2 is for Anatomical data. Similar types can be set for other data types}
            \PYG{n}{scans}\PYG{o}{.}\PYG{n}{download}\PYG{p}{(}\PYG{l+s+s2}{\PYGZdq{}}\PYG{l+s+s2}{/tmp}\PYG{l+s+s2}{\PYGZdq{}}\PYG{p}{,} \PYG{n+nb}{type}\PYG{o}{=}\PYG{l+s+s1}{\PYGZsq{}}\PYG{l+s+s1}{2}\PYG{l+s+s1}{\PYGZsq{}}\PYG{p}{,} \PYG{n}{extract}\PYG{o}{=}\PYG{n+nb+bp}{False}\PYG{p}{)}
        \PYG{k}{except}\PYG{p}{:}
            \PYG{k}{print} \PYG{l+s+s2}{\PYGZdq{}}\PYG{l+s+s2}{There are no scans to download}\PYG{l+s+s2}{\PYGZdq{}}
\end{sphinxVerbatim}


\chapter{Working with EEG data}
\label{\detokenize{Working_with_EEG_data:working-with-eeg-data}}\label{\detokenize{Working_with_EEG_data::doc}}
\begin{sphinxVerbatim}[commandchars=\\\{\}]
\PYG{k+kn}{import} \PYG{n+nn}{pyxnat}
\PYG{k+kn}{import} \PYG{n+nn}{os}

\PYG{c+c1}{\PYGZsh{} connect to XNAT instance}
\PYG{k+kn}{from} \PYG{n+nn}{pyxnat} \PYG{k+kn}{import} \PYG{n}{Interface}
\PYG{n}{xnat} \PYG{o}{=} \PYG{n}{Interface}\PYG{p}{(}\PYG{n}{server}\PYG{o}{=}\PYG{l+s+s1}{\PYGZsq{}}\PYG{l+s+s1}{http://sharp.bidmc.harvard.edu:8080}\PYG{l+s+s1}{\PYGZsq{}}\PYG{p}{,}  \PYG{n}{cachedir}\PYG{o}{=}\PYG{l+s+s1}{\PYGZsq{}}\PYG{l+s+s1}{/tmp}\PYG{l+s+s1}{\PYGZsq{}}\PYG{p}{)}
\PYG{n}{xnat}\PYG{o}{.}\PYG{n}{select}\PYG{o}{.}\PYG{n}{projects}\PYG{p}{(}\PYG{p}{)}\PYG{o}{.}\PYG{n}{get}\PYG{p}{(}\PYG{p}{)}
\end{sphinxVerbatim}

\begin{sphinxVerbatim}[commandchars=\\\{\}]
\PYG{n}{project} \PYG{o}{=} \PYG{n}{xnat}\PYG{o}{.}\PYG{n}{select}\PYG{o}{.}\PYG{n}{project}\PYG{p}{(}\PYG{l+s+s1}{\PYGZsq{}}\PYG{l+s+s1}{EEG\PYGZus{}SHARP}\PYG{l+s+s1}{\PYGZsq{}}\PYG{p}{)}
\end{sphinxVerbatim}

\begin{sphinxVerbatim}[commandchars=\\\{\}]
\PYG{n}{subject} \PYG{o}{=} \PYG{n}{project}\PYG{o}{.}\PYG{n}{subject}\PYG{p}{(}\PYG{l+s+s1}{\PYGZsq{}}\PYG{l+s+s1}{0001}\PYG{l+s+s1}{\PYGZsq{}}\PYG{p}{)}
\end{sphinxVerbatim}

\begin{sphinxVerbatim}[commandchars=\\\{\}]
\PYG{n}{experiments} \PYG{o}{=} \PYG{n}{subject}\PYG{o}{.}\PYG{n}{experiments}\PYG{p}{(}\PYG{p}{)}
\end{sphinxVerbatim}

\begin{sphinxVerbatim}[commandchars=\\\{\}]
\PYG{k}{for} \PYG{n}{list\PYGZus{}experiment} \PYG{o+ow}{in} \PYG{n}{experiments}\PYG{p}{:}
    \PYG{n}{files} \PYG{o}{=} \PYG{n}{list\PYGZus{}experiment}\PYG{o}{.}\PYG{n}{resources}\PYG{p}{(}\PYG{p}{)}\PYG{o}{.}\PYG{n}{files}\PYG{p}{(}\PYG{p}{)}
    \PYG{k}{for} \PYG{n}{fl} \PYG{o+ow}{in} \PYG{n}{files}\PYG{p}{:}
        \PYG{k}{print} \PYG{p}{(}\PYG{n}{fl}\PYG{p}{)}
\end{sphinxVerbatim}


\chapter{Definitions for Parameters Listed in the EEG DATA}
\label{\detokenize{Data_Definations_Phase1B::doc}}\label{\detokenize{Data_Definations_Phase1B:definitions-for-parameters-listed-in-the-eeg-data}}
All generated files are formatted as two dimensional tables. The rows
correspond to a subject activity such as an IQ test or a Switch task
problem, and the columns list data for the activity. This section
describes the parameters that appear in the table columns.

The parameter definitions are described in the following subsections:
\begin{itemize}
\item {} 
Parameters that are defined for many (even all) generated files, and
parameters that are specific to the following activities:
\begin{itemize}
\item {} 
The BOMAT and Sandia IQ tests

\item {} 
The Inhibit and Switch EF tasks

\item {} 
The Silo Detection, Thumbprint Detection and Visual Search Active
Control training

\end{itemize}

\item {} 
Parameters that are specific to the Rotation Span EF task

\item {} 
Parameters that are specific to RobotFactory training

\item {} 
Parameters that are specific to the questionnaires

\item {} 
Parameters that are specific to the EEG files: the blinded EEG data
files and the generated \sphinxstyleemphasis{eeg-sum} file

\end{itemize}

Unless noted otherwise, the time units for values extracted from
Presentation log files are expressed in units of 1/10$^{\text{th}}$ of a
millisecond (Presentation time).


\section{Parameters That Apply to Many of the Generated Files}
\label{\detokenize{Data_Definations_Phase1B:parameters-that-apply-to-many-of-the-generated-files}}

\begin{savenotes}\sphinxatlongtablestart\begin{longtable}{|*{2}{\X{1}{2}|}}
\hline
\sphinxstylethead{\sphinxstyletheadfamily 
\sphinxstyleemphasis{Parameter}
\unskip}\relax &\sphinxstylethead{\sphinxstyletheadfamily 
\sphinxstyleemphasis{Description}
\unskip}\relax \\
\hline
\endfirsthead

\multicolumn{2}{c}%
{\makebox[0pt]{\sphinxtablecontinued{\tablename\ \thetable{} -- continued from previous page}}}\\
\hline
\sphinxstylethead{\sphinxstyletheadfamily 
\sphinxstyleemphasis{Parameter}
\unskip}\relax &\sphinxstylethead{\sphinxstyletheadfamily 
\sphinxstyleemphasis{Description}
\unskip}\relax \\
\hline
\endhead

\hline
\multicolumn{2}{r}{\makebox[0pt][r]{\sphinxtablecontinued{Continued on next page}}}\\
\endfoot

\endlastfoot

Accuracy
&
The proportion of attempted problems that were solved correctly. For the Switch task, only switch cases (the cue has changed from the previous problem) are tallied.
\\
\hline
Age
&
The subject’s age expressed in years.
\\
\hline
AgeBin
&
The subject’s age expressed in one of these ranges:
\begin{itemize}
\item {} 
\textless{}21 (less than 21 years old)

\item {} 
21-25

\item {} 
26-30

\item {} 
31-35

\item {} 
36-40

\item {} 
41-45

\item {} 
46-50

\item {} 
51-55

\item {} 
56-60

\item {} 
61-65

\item {} 
66-70

\item {} 
\textgreater{}70 (older than 70)

\end{itemize}
\\
\hline
Animal
&
Used with the Inhibit task, set to TRUE when the stimulus represents an animal and FALSE if not.
\\
\hline
Att10
&
Used with the IQ tests, the number of problems answered by the subject during the first 10 minutes of the test. For the Sandia test, this value includes cases when the subject did not provide an answer within the 60 second per problem deadline.
\\
\hline
Att5
&
Used with the IQ tests, the number of problems answered by the subject in the first 5 minutes of the test. For the Sandia test, this value includes cases when the subject did not provide an answer within the 60 second per problem deadline.
\\
\hline
Attempted
&
The number of problems attempted and answered by the subject. For the Sandia test, this value includes cases when the subject did not provide an answer within the 60 second per problem deadline.
\\
\hline
Avg\sphinxstyleemphasis{i}
&
For an active control task (silo detection, thumbprint or visual search), the average difficulty level over the subject’s \sphinxstyleemphasis{i}$^{\text{th}}$ training session.
\\
\hline
AvgRT
&
The subject’s average response time for answering problems. For the Switch task, only switch cases (the cue has changed from the previous problem) are used in the computation.
\\
\hline
Block
&
The EF and active control tasks are divided into sections that are identified by a block label, e.g., PRACTICE, PERFORMANCE, etc.
\\
\hline
Condition
&The subject’s intervention protocol, a combination of:
\begin{itemize}
\item {} 
The type of training (RobotFactory or Active Control)

\item {} 
The type of tES

\end{itemize}

Values are:
\begin{itemize}
\item {} 
RF tDCS

\item {} 
RF tRNS

\item {} 
RF tDCS Sham

\item {} 
RF tRNS Sham

\item {} 
AC tDCS Sham

\item {} 
AC tRNS Sham

\end{itemize}
\\
\hline
Corr10
&
The number of IQ test problems solved correctly during the test’s first 10 minutes.
\\
\hline
Corr5
&
The number of IQ test problems solved correctly during the test’s first 5 minutes.
\\
\hline
Correct
&
The number of problems solved correctly. For the Switch task, only switch cases (the cue has changed from the previous problem) are tallied.
\\
\hline
CorrNoResp
&
An inhibit trial (inhibit case) was answered correctly by no keyboard response.
\\
\hline
CorrResp
&
An inhibit trial (don’t inhibit case) was answered correctly by the correct keyboard response.
\\
\hline
Cue
&
The switch cue that was presented, either a heart or a cross.
\\
\hline
CueTime
&
The Presentation time when the switch cue was presented.
\\
\hline
CueUncert
&
The Presentation-computed uncertainty in the CueTime.
\\
\hline
Date
&
The date when the data file was created. With the exception of the side-effects questionnaires, this is also the date the subject performed the activity.
\\
\hline
Delay
&
For the Switch task, the duration between the presentation of the cue and the presentation of the stimulus, expressed in Presentation time units.
\\
\hline
Duration
&
The elapsed time for subject to complete the active control task.
\\
\hline
EduLevel
&
The level of education achieved by the subject expressed as one of:
\begin{itemize}
\item {} 
no high school

\item {} 
some high school

\item {} 
high school graduate

\item {} 
some college

\item {} 
college graduate

\item {} 
some master’s degree or higher

\item {} 
completed master’s degree or higher

\end{itemize}
\\
\hline
EduYears
&
The subject’s educational level expressed in years. For example, undergrad completion is typically 16 years.
\\
\hline
Expected
&
The correct response to an active control task problem (left or right shift key).
\\
\hline
Final\sphinxstyleemphasis{i}
&
For an active control task (silo detection, thumbprint or visual search), the subject’s difficulty level and the end of the \sphinxstyleemphasis{i}$^{\text{th}}$ training session.
\\
\hline
Gender
&
The subject’s gender, male or female.
\\
\hline
Ho/He
&
Used with the Visual Search task to distinguish between the easier homogenous problems (the same character used for all distracters) from the more difficult heterogeneous problems (many characters are used for the distracters).
\\
\hline
IgnoredEvents
&
Ignored events occur when the subject presses a keyboard key during an EF or AC task when a subject response is not expected. Excessively high numbers during a task could suggest a problem with the keyboard or a subject’s lack of cooperation doing the task.

In summary files, this parameter expresses the number of ignored events that occurred during the task or test. In detailed files, it consists of a letter followed by a number. The letter indicates the key pressed (P =\textgreater{} subject paused the scenario, R =\textgreater{} subject resumed the scenario, X =\textgreater{} all other keys). The number is the Presentation time for the event.
\\
\hline
IncorrNoResp
&
An inhibit trial (inhibit case) was answered incorrectly by a keyboard response.
\\
\hline
IncorrResp
&
An inhibit trial (don’t inhibit case) was answered with the incorrect keyboard response.
\\
\hline
Init\sphinxstyleemphasis{i}
&
For an active control task (silo detection, thumbprint or visual search), the difficulty level at the start of the subject’s \sphinxstyleemphasis{i}$^{\text{th}}$ training session.
\\
\hline
InhCorr
&This field captures information about the subject’s response when an inhibit cue is presented. It takes one of these four values:
\begin{itemize}
\item {} 
True \textendash{} inhibit cue was presented and subject did not press a shift key

\item {} 
Before \textendash{} inhibit cue was presented and subject had already pressed a shift key before the cue was presented

\item {} 
After \textendash{} inhibit cue was presented and subject pressed a shift key after the cue was presented

\item {} 
Blank \textendash{} an inhibit cue was not presented

\end{itemize}

The parser does not account for human reaction time. For example, if a key press occurs one millisecond after the cue, the InCorr value will be set to “After” even though from the perspective of the human’s response time, the “Before” value might be considered more appropriate.
\\
\hline
Inhibit
&
The duration before the sounding of the inhibit cue:
\begin{itemize}
\item {} 
S50 \textendash{} the inhibit cue was presented 50 ms after presenting the noun

\item {} 
NNN \textendash{} the inhibit cue was presented NNN ms after presenting the noun, where NNN is computed by a staircase algorithm that considers the response times of the subject’s previous keyboard presses

\item {} 
Blank \textendash{} an inhibit cue was not presented

\end{itemize}
\\
\hline
InhTime
&
The Presentation time when the inhibit cue was presented; blank when inhibit cue is not presented.
\\
\hline
InhUncert
&
Presentation computed uncertainty associated with InhTime.
\\
\hline
Institution
&
The institution that performed the trial, Harvard, Honeywell, Northeastern (NEU) or Oxford.
\\
\hline
IsLogical
&
Used with the Sandia test to denote a “logical” problem. The Sandia test consists of logical and relational problems.
\\
\hline
IsSwitch
&
Set to TRUE for switch problems where the cue has changed since the previous problem. This condition will be true for approximately half of all switch problems.
\\
\hline
Large
&
True if the stimulus noun represents something bigger than a soccer ball, and false otherwise.
\\
\hline
LevelMax
&
The most difficult level for the task attempted by the subject over the course of an active control training session.
\\
\hline
LevelMin
&
The easiest level for the task attempted by the subject over the course of an active control training session.
\\
\hline
Living
&
True if the stimulus noun represents a living entity, and false otherwise.
\\
\hline
LogAtt
&
The number of Sandia test logical problems attempted by the subject.
\\
\hline
LogAtt10
&
The number of Sandia logical problems attempted by the subject during the first 10 minutes of the test.
\\
\hline
LogAtt5
&
The number of Sandia logical problems attempted by the subject during the first 5 minutes of the test.
\\
\hline
LogCorr
&
The number of Sandia test logical problems solved correctly by the subject.
\\
\hline
LogCorr10
&
The number of Sandia logical problems solved correctly by the subject during the first 10 minutes of the test.
\\
\hline
LogCorr5
&
The number of Sandia logical problems solved correctly by the subject during the first 5 minutes of the test.
\\
\hline
LogTO
&
The number of times the subject reached the 60-second Sandia problem time limit while solving a logical test problem.
\\
\hline
LogTO10
&
During the first 10 minutes of a Sandia test, the number of times the subject reached the 60-second problem time limit while solving a logical test problem.
\\
\hline
LogTO5
&
During the first 5minutes of a Sandia test, the number of times the subject reached the 60-second problem time limit while solving a logical test problem.
\\
\hline
Max\sphinxstyleemphasis{i}
&
For an active control task (silo detection, thumbprint or visual search), the maximum difficulty level for the subject’s \sphinxstyleemphasis{i}$^{\text{th}}$ training session.
\\
\hline
Min\sphinxstyleemphasis{i}
&
For an active control task (silo detection, thumbprint or visual search), the minimum difficulty level for the subject’s \sphinxstyleemphasis{i}$^{\text{th}}$ training session.
\\
\hline
NAccuracy
&
Subject’s accuracy solving non-switch problems (switch cue is unchanged from the previous problem).
\\
\hline
NAvgRT
&
Used for the Switch task, the subject’s average response time solving non-switch problems (switch cue is unchanged from the previous problem).
\\
\hline
NCorrect
&
The number of non-switch problems (switch cue is unchanged from the previous problem) that the subject solved correctly.
\\
\hline
NonSwitch
&
Count of non-switch problems attempted by the subject (the switch cue is unchanged from the previous problem).
\\
\hline
NTimeout
&
The number of timeouts that occurred when the subject was solving a non-switch problem (switch cue is unchanged from the previous problem).
\\
\hline
Period
&
The trial phase period:
\begin{itemize}
\item {} 
Pre-test

\item {} 
Training

\item {} 
Post-test

\end{itemize}
\\
\hline
Problem
&
The problem number.
\\
\hline
RelAtt
&
The number of Sandia test relational problems attempted by the subject.
\\
\hline
RelAtt10
&
During the first 10 minutes of the test, the number of Sandia relational problems attempted by the subject.
\\
\hline
RelAtt5
&
During the first 5 minutes of the test, the number of Sandia relational problems attempted by the subject during the first 5 minutes of the test.
\\
\hline
RelCorr
&
The number of Sandia test relational problems solved correctly by the subject.
\\
\hline
RelCorr10
&
During the first 10 minutes of the test, the number of Sandia relational problems solved correctly by the subject.
\\
\hline
RelCorr5
&
During the first 5 minutes of the test, the number of Sandia relational problems solved correctly by the subject.
\\
\hline
RelTO
&
The number of times the subject reached the 60 second Sandia problem time limit while solving a relational test problem.
\\
\hline
RelTO10
&
During the first 10 minutes of the test, the number of times the subject reached the 60 second Sandia problem time limit while solving a relational test problem.
\\
\hline
RelTO5
&
During the first 5 minutes of the test, the number of times the subject reached the 60 second Sandia problem time limit while solving a relational test problem.
\\
\hline
Response
&
The subject’s response. For the IQ tests it is a number (1..6 for BOMAT and Ravens and 1..8 for Sandia). For the Inhibit, Switch and active control tasks, the response is a either the left-shift key or the right-shift key.
\\
\hline
RespTime
&
The Presentation time when the subject entered a response.
\\
\hline
RespUncert
&
The Presentation-computed uncertainty in the subject’s response time.
\\
\hline
Score
&
The subject’s score for the problem, true if correct, false if incorrect, or timeout.
\\
\hline
Silos
&
In the Silo Detection task, the number of silos in the image presented to subject.
\\
\hline
Status
&
The subject’s completion status, Active (not completed), Finished, Excluded, Quit or Dropped. Subjects marked as Finished completed the intervention. Subjects marked as Excluded completed the intervention but were excluded from the analysis, generally because they showed less than a minimum threshold of participation during the intervention. The Quit and Dropped status indicate that the subject did not complete the intervention, either because the subject quit or because the experimenters had to drop the subject due to missed appointments, was later found to have an exclusionary condition, etc.
\\
\hline
StimTime
&
Time when the stimulus was presented to the subject
\\
\hline
Stimulus
&
The stimulus presented to the subject.
\\
\hline
StimUncert
&
The Presentation-computed uncertainty in the stimulation time.
\\
\hline
Subject

SubjNum
&
A unique subject identifier. Either the (unblinded to experimenters) MITRE participant ID number without the “HON” prefix, or a blinded subject id. Blinded subject ids are randomly assigned from the range @0001-@0999.
\\
\hline
SubjWord
&
The subject’s unblinded unique mnemonic id.
\\
\hline
Timeout
&
For the Sandia test, the number of times the 60 second per problem timer expired without a subject response. For the Switch task, the number of times the subject did not respond to a switch problem (the cue has changed from the previous problem).
\\
\hline
TO10
&
During the first 10 minutes of a Sandia test, the number of times the 60 second per problem timer expired without a subject response.
\\
\hline
TO5
&
During the first 5 minutes of a Sandia test, the number of times the 60 second per problem timer expired without a subject response.
\\
\hline
Upload
&
The week during which the data was added to the parser’s database. The upload date is typically a Monday but the actual upload may have been a few days before or after. Most often the data was recorded during the previous week.
\\
\hline
\end{longtable}\sphinxatlongtableend\end{savenotes}


\begin{savenotes}\sphinxattablestart
\centering
\begin{tabulary}{\linewidth}[t]{|T|}
\hline
\\
\hline
\end{tabulary}
\par
\sphinxattableend\end{savenotes}


\section{Parameters That Are Specific to the Rotation Span Executive Function Task}
\label{\detokenize{Data_Definations_Phase1B:parameters-that-are-specific-to-the-rotation-span-executive-function-task}}

\begin{savenotes}\sphinxatlongtablestart\begin{longtable}{|*{2}{\X{1}{2}|}}
\hline
\sphinxstylethead{\sphinxstyletheadfamily 
\sphinxstyleemphasis{Parameter}
\unskip}\relax &\sphinxstylethead{\sphinxstyletheadfamily 
\sphinxstyleemphasis{Description}
\unskip}\relax \\
\hline
\endfirsthead

\multicolumn{2}{c}%
{\makebox[0pt]{\sphinxtablecontinued{\tablename\ \thetable{} -- continued from previous page}}}\\
\hline
\sphinxstylethead{\sphinxstyletheadfamily 
\sphinxstyleemphasis{Parameter}
\unskip}\relax &\sphinxstylethead{\sphinxstyletheadfamily 
\sphinxstyleemphasis{Description}
\unskip}\relax \\
\hline
\endhead

\hline
\multicolumn{2}{r}{\makebox[0pt][r]{\sphinxtablecontinued{Continued on next page}}}\\
\endfoot

\endlastfoot

Arrow\sphinxstyleemphasis{i}
&
The \sphinxstyleemphasis{i}$^{\text{th}}$ arrow presented to the subject. Arrows are represented by a two character compass direction. The short arrows are expressed in lower case, the long arrows in upper case. For example, “nn” is the north (up) pointing short arrow and “NE” is the northeast pointing long arrow.
\\
\hline
ArrAns\sphinxstyleemphasis{i}
&
The subject’s answer for the \sphinxstyleemphasis{i}$^{\text{th}}$ arrow.
\\
\hline
ArrScore\sphinxstyleemphasis{i}
&
The subject’s score for answering the \sphinxstyleemphasis{i:sup:{}`th{}`} arrow.
\\
\hline
ArrPresentTime\sphinxstyleemphasis{i}
&
The Presentation time when the \sphinxstyleemphasis{i:sup:{}`th{}`} arrow was presented.
\\
\hline
ArrAnsTime\sphinxstyleemphasis{i}
&
The Presentation time when the subject answered the \sphinxstyleemphasis{i:sup:{}`th{}`} arrow.
\\
\hline
ArrowsAcc
&
Overall accuracy of subject’s arrow recall.
\\
\hline
ArrowsBlank
&
Number of arrows marked as “blank” during the arrow recall phase.
\\
\hline
ArrowsCleared
&
Number of arrows that were cleared during the arrow recall phase.
\\
\hline
ArrowsClearedCmds
&
Number of times arrows were cleared during the arrow recall phase.
\\
\hline
ArrowsCorr
&
The number of arrows that were recalled correctly during the arrow recall phase.
\\
\hline
ArrowsExtra
&
Number of extra arrows provided during arrow recall phase (e.g., four arrows were presented and the subject’s recall contains more than four arrows).
\\
\hline
ArrowsMissing
&
Number of arrows that were not provided during arrow recall phase (e.g., four arrows were presented and the subject’s recall contains fewer than four arrows).
\\
\hline
ClearedArrows
&
Lists of recalled arrow sequences that were cleared (if any) before the subject submitted his response. Each sequence begins with a timestamp for the start of the recall which is followed by the list of arrows being cleared and then terminated by the time when the sequence was cleared.
\\
\hline
ClearedEndTime
&
If subject cleared arrows during arrow recall, the Presentation time of the last clearance (hence the starting time for recalling the arrows that were not cleared).
\\
\hline
DistractorDwellTimeLimit (ms)
&The maximum time allowed for the subject to respond to the letter query (“was the letter normal or inverted?”). The number is computed when the subject performs the pretest version of the task as:
\begin{quote}

\sphinxstyleemphasis{average response time + 3 * stdev response time}
\end{quote}

for the letters answered correctly during a letter practice session. The value computed during pretest is reused during posttest.
\\
\hline
ExtraArrows
&
The list of extra arrows for cases when the subject’s arrow response contains more arrows than were presented to the subject. Two values are provided for each response, the time the entry was made and the identity of the extra arrow.
\\
\hline
Letter\sphinxstyleemphasis{i}
&The \sphinxstyleemphasis{i}$^{\text{th}}$ letter presented to the subject. Letters are described in these three parts:
\begin{itemize}
\item {} 
The letter displayed (‘F’, ‘G’, ‘J’, or ‘R’)

\item {} 
A rotation from the unrotated position (one of eight compass points expressed by two character mnemonic \textendash{} NN, NE, EE, SE, SS, SW, WW, NW) where NN is considered to be the unrotated position

\item {} 
The symbol ‘\textbar{}\textgreater{}’ if presented in normal orientation and ‘\textless{}\textbar{}’ if presented inverted.

\end{itemize}

So for example ‘G EE \textless{}\textbar{}’ indicates that the letter G was presented rotated 90 degrees clockwise and then inverted.
\\
\hline
LetScore\sphinxstyleemphasis{i}
&
The subject’s response to the \sphinxstyleemphasis{i}$^{\text{th}}$ letter (inverted or not).
\\
\hline
LetPresentTime\sphinxstyleemphasis{i}
&
The Presentation time when the \sphinxstyleemphasis{i}$^{\text{th}}$ letter was presented.
\\
\hline
LetDismissTime\sphinxstyleemphasis{i}
&
The Presentation time when the subject dismissed the \sphinxstyleemphasis{i}$^{\text{th}}$ letter.
\\
\hline
LetQueryTime\sphinxstyleemphasis{i}
&
The Presentation time when the subject was prompted for a response to the \sphinxstyleemphasis{i}$^{\text{th}}$ letter.
\\
\hline
LetAnsTime\sphinxstyleemphasis{i}
&
The Presentation time when the subject responded to the \sphinxstyleemphasis{i}$^{\text{th}}$ letter.
\\
\hline
LettersAcc
&
Accuracy of the letter responses (is letter inverted or not?).
\\
\hline
LettersCorr
&
The number of letters answered correctly.
\\
\hline
LettersTO
&
The number of letter presentations that ended without a subject response.
\\
\hline
NumArrBlank
&
Number of arrows that were set to “blank” instead of being recalled.
\\
\hline
NumArrCleared
&
The number of arrows that were cleared by arrow clear commands.
\\
\hline
NumArrCorr
&
Number of arrows that were recalled correctly.
\\
\hline
NumArrExtra
&
Number of extra arrows in the arrow recalls.
\\
\hline
NumArrMissing
&
Number of arrows that were missing from the arrow recalls.
\\
\hline
NumLetCorrect
&
The number of letters answered correctly (normal or inverted?)
\\
\hline
NumLetTimeoutPresent
&
Number of times the subject failed to dismiss the letter within the permitted letter presentation interval.
\\
\hline
NumLetTimeoutResp
&
Number of times the subject failed to answer the letter query (normal or inverted) within the query presentation period.
\\
\hline
NumTimesArrCleared
&
Number of times that the subject restarted the arrow recall by erasing the currently recalled arrows.
\\
\hline
RecallBeginTime
&
The time when the subject began the arrow recall sequence.
\\
\hline
SeqBeginTime
&
The time when the subject started the letter/arrow sequence.
\\
\hline
SeqEndTime
&
The time when the subject completed the recall sequence.
\\
\hline
SeqLength
&
The number of letter arrow pairs in the Rotation Span sequence (3, 4 or 5).
\\
\hline
Trials
&
The number of trials performed by the subject. The full rotation span task contains 70 trials.

Note: A bug in the Rotation Span scenario that was not resolved until midway through phase 1B testing caused the log file for a single Rotation Span sequence (3, 4 or 5 letter/arrow pairs) to be incomplete. The parser does not attempt to score these incomplete sequences. This error occurred 13 times, and for those instances the “rotation-sum” file will show 65, 66, or 67 in the “Trials” column instead of the expected 70.
\\
\hline
\end{longtable}\sphinxatlongtableend\end{savenotes}


\begin{savenotes}\sphinxattablestart
\centering
\begin{tabulary}{\linewidth}[t]{|T|}
\hline
\\
\hline
\end{tabulary}
\par
\sphinxattableend\end{savenotes}


\section{Parameters That Are Specific to RobotFactory}
\label{\detokenize{Data_Definations_Phase1B:parameters-that-are-specific-to-robotfactory}}
{[}TBD: need inputs from RobotFactory developers.{]}


\begin{savenotes}\sphinxatlongtablestart\begin{longtable}{|l|l|}
\hline
\sphinxstylethead{\sphinxstyletheadfamily 
\sphinxstyleemphasis{Parameter}
\unskip}\relax &\sphinxstylethead{\sphinxstyletheadfamily 
\sphinxstyleemphasis{Description}
\unskip}\relax \\
\hline
\endfirsthead

\multicolumn{2}{c}%
{\makebox[0pt]{\sphinxtablecontinued{\tablename\ \thetable{} -- continued from previous page}}}\\
\hline
\sphinxstylethead{\sphinxstyletheadfamily 
\sphinxstyleemphasis{Parameter}
\unskip}\relax &\sphinxstylethead{\sphinxstyletheadfamily 
\sphinxstyleemphasis{Description}
\unskip}\relax \\
\hline
\endhead

\hline
\multicolumn{2}{r}{\makebox[0pt][r]{\sphinxtablecontinued{Continued on next page}}}\\
\endfoot

\endlastfoot

ActualN
&\\
\hline
Automaton
&\\
\hline
AvgRespTimeEst
&\\
\hline
BaseStopSignalDelay
&\\
\hline
Cluster
&
The RF cluster associated with the shift.
\\
\hline
CurrentAccuracy
&\\
\hline
ClusterComplete
&\\
\hline
CurrentLevel
&\\
\hline
CurrentStopSignalDelay
&\\
\hline
CurrentTask
&\\
\hline
DisplayedInhSignal
&\\
\hline
EstRespTime
&\\
\hline
ExpResp
&\\
\hline
GivenResp
&\\
\hline
InhibitDelayUsed
&\\
\hline
InhibitProb
&\\
\hline
isLastDay
&
When set to TRUE, indicates that the subject received the special experience provided for the last training session.
\\
\hline
JumpBack
&\\
\hline
LastMedianResponseTime
&\\
\hline
Level
&
The game’s level of difficulty.
\\
\hline
LogFileId
&\\
\hline
MatchProb
&\\
\hline
N
&\\
\hline
n-Back
&\\
\hline
n-BackProb
&\\
\hline
NextInhibitDelay
&\\
\hline
NextState
&\\
\hline
PassCount
&\\
\hline
PreInhExpResp
&\\
\hline
PreInhNextState
&\\
\hline
PreInhSigmaP
&\\
\hline
PreInhSigmaPRight
&\\
\hline
ProgressionDataFile
&\\
\hline
ReportedRespTime
&\\
\hline
RespTime
&\\
\hline
Shift
&
The name of the RobotFactory game being played during a two minute shift.
\\
\hline
ShiftNum
&
A count of the two-minute RobotFactory shifts played by the subject.
\\
\hline
ShortSsdChance
&\\
\hline
ShortStopSignalDelay
&\\
\hline
ShortStopSignalDelayChance
&\\
\hline
SrtEstimateConstAlpha
&\\
\hline
StopSignalDelayStepValue
&\\
\hline
SigmaP
&\\
\hline
SigmaPRight
&\\
\hline
SigmaS
&\\
\hline
SigmaS\_Color
&\\
\hline
SigmaS\_Grid
&\\
\hline
SigmaS\_Number
&\\
\hline
SigmaS\_Picture
&\\
\hline
SigmaS\_Shape
&\\
\hline
SigmaS\_Word
&\\
\hline
SigmaSR\_Color
&\\
\hline
SigmaSR\_Grid
&\\
\hline
SigmaSR\_Number
&\\
\hline
SigmaSR\_Picture
&\\
\hline
SigmaSR\_Shape
&\\
\hline
SigmaSR\_Word
&\\
\hline
SigmaSRight
&\\
\hline
StimShowTime
&\\
\hline
SubjLastMedianRT
&\\
\hline
SubjSSRT
&\\
\hline
SwitchProb
&\\
\hline
UpdatedInhDelay
&\\
\hline
TrialId
&
A cross-reference into the rf-triggers file used by EEG analysis tools.
\\
\hline
TrialTime
&\\
\hline
UsedShortDelay
&\\
\hline
\end{longtable}\sphinxatlongtableend\end{savenotes}


\section{Parameters That Are S­pecific to the Questionnaires}
\label{\detokenize{Data_Definations_Phase1B:parameters-that-are-specific-to-the-questionnaires}}
Parameters specific to each questionnaire are listed in the following
tables in the order they appear in the log files.


\subsection{Demographic Questionnaire}
\label{\detokenize{Data_Definations_Phase1B:demographic-questionnaire}}

\begin{savenotes}\sphinxattablestart
\centering
\begin{tabulary}{\linewidth}[t]{|T|T|T|T|T|T|T|}
\hline
&&&&&&\\
\hline
\end{tabulary}
\par
\sphinxattableend\end{savenotes}


\begin{savenotes}\sphinxattablestart
\centering
\begin{tabulary}{\linewidth}[t]{|T|T|}
\hline
\sphinxstylethead{\sphinxstyletheadfamily 
\sphinxstyleemphasis{Parameter}
\unskip}\relax &\sphinxstylethead{\sphinxstyletheadfamily 
\sphinxstyleemphasis{Question}
\unskip}\relax \\
\hline
Sex
&
Sex (Female or Male)
\\
\hline
AgeQ
&
Age
\\
\hline
Weight
&
Weight expressed in pounds
\\
\hline
Height
&
Height expressed in inches
\\
\hline
Race/Ethnicity
&
Race/Ethnicity (select all that apply): White, African American, Hispanic or Latino, Asian, American Indian/Alaskan Native, Native Hawaiian/Pacific Islander, Unknown, Other {[}specify{]}
\\
\hline
Cigarettes
&
Do you smoke cigarettes? (No or Yes)
\\
\hline
NumCigs
&
If Yes, how many cigarettes do you typically smoke per day?
\\
\hline
TBI
&
Have you ever had a diagnosed concussion or mild traumatic brain injury? (No or Yes)
\\
\hline
TbiAge
&
If yes, at what age?
\\
\hline
Unconscious
&
If yes, did you lose consciousness? (Yes or No)
\\
\hline
UnconTime
&
If yes, for how long did you lose consciousness (minutes)?
\\
\hline
EngLangAge
&
At what age did you begin learning English? If English is your native language that you were exposed to from birth, respond with 0.
\\
\hline
Languages
&
Please specify which language(s), other than English, you speak (if any) (separate each language with a comma).
\\
\hline
LangAges
&
Ages Learned (enter as number \& separate each language with a comma).
\\
\hline
Major
&
Major area of study in college. Leave blank if you did not attend college.
\\
\hline
EduLevelQ
&
Highest level of education (self) (no high school, some high school, high school graduate, some college, college graduate, some master’s degree or higher, completed master’s degree or higher).
\\
\hline
EduYearsQ
&
Total years of education - enter as numeral (example, undergrad completion is typically 16 years).
\\
\hline
EduLevelMother
&
Highest level of education (mother).
\\
\hline
EduLevelFather
&
Highest level of education (father).
\\
\hline
Occupation
&
Occupation (self).
\\
\hline
OccMother
&
Occupation (mother).
\\
\hline
OccFather
&
Occupation (father).
\\
\hline
VideoTime
&
On average, how many hours a week do you play video/computer games?
\\
\hline
VideoGames
&
How many different video/computer games do you typically play in a year?
\\
\hline
\end{tabulary}
\par
\sphinxattableend\end{savenotes}


\subsection{Physical Activity Questionnaire \#1 (Godin)}
\label{\detokenize{Data_Definations_Phase1B:physical-activity-questionnaire-1-godin}}

\begin{savenotes}\sphinxattablestart
\centering
\begin{tabular}[t]{|*{2}{\X{1}{2}|}}
\hline
\sphinxstylethead{\sphinxstyletheadfamily 
\sphinxstyleemphasis{Parameter}
\unskip}\relax &\sphinxstylethead{\sphinxstyletheadfamily 
\sphinxstyleemphasis{Question}
\unskip}\relax \\
\hline
Level
&
Choose ONE activity category that best describes your usual pattern of daily physical activities, including activities related to house and family care, transportation, occupation, exercise and wellness, and leisure or recreational purposes.
\begin{itemize}
\item {} 
Level 1: Inactive or little activity other than usual daily activities.

\item {} 
Level 2: Regularly (≥5 days/wk) participate in physical activities requiring low levels of exertion that result in slight increases in breathing and heart rate for at least 10 MINUTES at a time.

\item {} 
Level 3: Participate in aerobic exercises such as brisk walking, jogging or running, cycling, swimming or vigorous sports at a comfortable pace or other activities requiring similar levels of exertion for 20 to 60 MINUTES per week.

\item {} 
Level 4: Participate in aerobic exercises such as brisk walking, jogging or running at a comfortable pace, or other activities requiring similar levels of exertion for 1 to 3 HOURS per week.

\item {} 
Level 5: Participate in aerobic exercises such as brisk walking, jogging, or running at a comfortable pace, or other activities requiring similar levels of exertion for OVER 3 HOURS per week.

\end{itemize}
\\
\hline
Strenuous
&
During a typical 7-day period (a week), how many times on average do you do the following kinds of exercise for more than 15 minutes during your free time?
\begin{enumerate}
\item {} 
STRENUOUS EXERCISE (heart beats rapidly) e.g., running, jogging, hockey, football, soccer, squash, basketball, cross country skiing, judo, roller skating, vigorous swimming, vigorous long distance bicycling

\end{enumerate}
\\
\hline
Moderate
&\begin{enumerate}
\item {} 
MODERATE EXERCISE (not exhausting) e.g., fast walking, baseball, tennis, easy bicycling, volleyball, badminton, easy swimming, alpine skiing, popular and folk dancing

\end{enumerate}
\\
\hline
Mild
&\begin{enumerate}
\item {} 
MILD EXERCISE (minimal effort) e.g., yoga, archery, fishing from river bank, bowling, horseshoes, golf, snow-mobiling, easy walking

\end{enumerate}
\\
\hline
Sweat
&
During a typical 7-day period (a week), in your leisure time, how often do you engage in any regular activity long enough to work up a sweat (heart beats rapidly)? (Often, Sometimes, Never/Rarely)
\\
\hline
\end{tabular}
\par
\sphinxattableend\end{savenotes}


\begin{savenotes}\sphinxattablestart
\centering
\begin{tabulary}{\linewidth}[t]{|T|T|T|T|T|}
\hline
&&&&\\
\hline
\end{tabulary}
\par
\sphinxattableend\end{savenotes}


\subsection{Physical Activity Questionnaire \#2 (MAQ)}
\label{\detokenize{Data_Definations_Phase1B:physical-activity-questionnaire-2-maq}}

\begin{savenotes}\sphinxattablestart
\centering
\begin{tabular}[t]{|*{2}{\X{1}{2}|}}
\hline
\sphinxstylethead{\sphinxstyletheadfamily 
\sphinxstyleemphasis{Parameter}
\unskip}\relax &\sphinxstylethead{\sphinxstyletheadfamily 
\sphinxstyleemphasis{Question}
\unskip}\relax \\
\hline
Activities
&
Please check the box next to all activities listed below that you have done more than 10 times in the past year:
\begin{itemize}
\item {} 
Jogging (outdoor, treadmill)

\item {} 
Swimming (laps, snorkeling)

\item {} 
Bicycling (indoor, outdoor)

\item {} 
Softball/Baseball

\item {} 
Volleyball

\item {} 
Bowling

\item {} 
Basketball

\item {} 
Skating (roller, ice, blading)

\item {} 
Martial Arts (karate, judo)

\item {} 
Tai Chi

\item {} 
Calisthenics/Toning exercises

\item {} 
Wood Chopping

\item {} 
Water/coal hauling

\item {} 
Football/Soccer

\item {} 
Racquetball/Handball/Squash

\item {} 
Horseback riding

\item {} 
Hunting

\item {} 
Fishing

\item {} 
Aerobic Dance/Step Aerobic

\item {} 
Water Aerobics

\item {} 
Dancing (Square, Line, Ballroom)

\item {} 
Gardening or Yardwork

\item {} 
Badminton

\item {} 
Strength/Weight training

\item {} 
Rock climbing

\item {} 
Scuba diving

\item {} 
Stair Master

\item {} 
Fencing

\item {} 
Hiking

\item {} 
Tennis

\item {} 
Golf

\item {} 
Canoeing/Rowing/Kayaking

\item {} 
Water skiing

\item {} 
Jumping rope

\item {} 
Snow skiing (X-country/Nordic track)

\item {} 
Snow skiing (downhill)

\item {} 
Snow shoeing

\item {} 
Yoga

\item {} 
Walking for exercise (out/indoor, treadmill)

\item {} 
Other

\end{itemize}
\\
\hline
Jogging
&
For each activity that you checked above, check the button underneath the months you did each activity over the past year (12 months) and then estimate the average amount of time spent in that activity.
\\
\hline
JoggingTimes
&
Avg \# of times per month
\\
\hline
JoggingMins
&
Average \# of minutes each time
\\
\hlineAbove pattern shown above for Jogging:
\begin{itemize}
\item {} 
Jogging

\item {} 
JoggingTimes

\item {} 
JoggingMins

\end{itemize}

is repeated for each of the items listed on right.
&\begin{itemize}
\item {} 
Swimming

\item {} 
Dancing

\item {} 
Bicycling

\item {} 
Gardening

\item {} 
Baseball

\item {} 
Badminton

\item {} 
Volleyball

\item {} 
WeightTraining

\item {} 
Bowling

\item {} 
RockClimbing

\item {} 
Basketball

\item {} 
ScubaDiving

\item {} 
Skating

\item {} 
StairMaster

\item {} 
MartialArts

\item {} 
Fencing

\item {} 
TaiChi

\item {} 
Hiking

\item {} 
Calisthenics

\item {} 
Tennis

\item {} 
WoodChopping

\item {} 
Golf

\item {} 
Hauling

\item {} 
Canoeing

\item {} 
Football

\item {} 
WaterSkiing

\item {} 
Squash

\item {} 
JumpRope

\item {} 
Horseback

\item {} 
NordicSkiing

\item {} 
Hunting

\item {} 
DownhillSkiing

\item {} 
Fishing

\item {} 
SnowShoeing

\item {} 
AerobicsDance

\item {} 
Yoga

\item {} 
AerobicsWater

\item {} 
Walking

\item {} 
Other

\end{itemize}
\\
\hline
TV
&
In general, how many HOURS per DAY do you usually spend watching television?
\\
\hline
Confined?
&
Over this past year, have you spent more than one week confined to a bed or a chair as a result of an injury, illness, or surgery?
\\
\hline
ConfinedWeeks
&
How many weeks over this past year were you confined to a bed or chair?
\\
\hline
DiffOutOfBed?
&
Do you have difficulty doing any of the following activities?
\begin{itemize}
\item {} 
Getting in or out of a bed or chair? (No or Yes)

\end{itemize}
\\
\hline
DiffWalkAcrossRoom?
&\begin{itemize}
\item {} 
Walking across a small room without resting? (No or Yes)

\end{itemize}
\\
\hline
DiffWalk10Min?
&\begin{itemize}
\item {} 
Walking for 10 minutes without resting? (No or Yes)

\end{itemize}
\\
\hline
TeamSport
&
Did you ever compete in an individual or team sport (not including any time spent in sports performed during school physical education classes)? (No or Yes)
\\
\hline
TeamYears
&
How many total years did you participate?
\\
\hline
RecentJob
&
In the past calendar year (i.e., previous 365 days), have you had a job for more than one month? (No or yes)
\\
\hline&
(EXPERIMENTER should complete with participant)

List all the JOBS that you held over the past year for more than one month. Account for all 12 months of the past year. If unemployed/disabled/retired/homemaker/student during all or part of the past year, list as such and probe for job activities of a normal 8 hour day, 5 day week.
\\
\hline
JobName1
&
(text field)
\\
\hline
JobCode1
&Select from:

Not employed outside of the home:
\begin{enumerate}
\item {} 
Student

\item {} 
Home Maker

\item {} 
Retired

\item {} 
Disabled

\item {} 
Unemployed

\end{enumerate}

Employed (or volunteer):
\begin{enumerate}
\setcounter{enumi}{5}
\item {} 
Armed Services

\item {} 
Office Worker

\item {} 
Non-office Worker

\end{enumerate}
\\
\hline
Min/Day1
&
Walk or bicycle to/from work.
\\
\hline
Mos/Yr1
&\\
\hline
Days/Wk1
&
Average job schedule
\\
\hline
Hrs/Day1
&
Average job schedule
\\
\hline
HrsSitting1
&
Hours spent sitting at work
\\
\hline
Category1
&
Check the category that best describes job activities when not sitting:
\begin{itemize}
\item {} 
Category A (includes all sitting activities):
\begin{itemize}
\item {} 
Sitting

\item {} 
Standing still w/o heavy lifting

\item {} 
Light cleaning - ironing, cooking, washing, dusting

\item {} 
Driving a bus, taxi, tractor

\item {} 
Jewelry making/weaving

\item {} 
General office work

\item {} 
Occasional/short distance walking

\end{itemize}

\item {} 
Category B (includes most indoor activities):
\begin{itemize}
\item {} 
Carrying light loads

\item {} 
Continuous walking

\item {} 
Heavy cleaning - mopping, sweeping, scrubbing, vacuuming

\item {} 
Gardening - planting, weeding

\item {} 
Painting/Plastering

\item {} 
Plumbing/Welding

\item {} 
Electrical work

\item {} 
Sheep herding

\end{itemize}

\item {} 
Category C (heavy industrial work, outdoor construction, farming):
\begin{itemize}
\item {} 
Carrying moderate to heavy loads

\item {} 
Heavy construction

\item {} 
Farming - hoeing, digging, mowing, raking

\item {} 
Digging ditches, shoveling

\item {} 
Chopping (ax), sawing wood

\item {} 
Tree/pole climbing

\item {} 
Water/coal/wood hauling

\end{itemize}

\end{itemize}
\\
\hline
The JobName1..Category1 pattern repeats for up to six more jobs (JobName7-Category7).
&\\
\hline
\end{tabular}
\par
\sphinxattableend\end{savenotes}


\subsection{Acute Side Effects Questionnaire}
\label{\detokenize{Data_Definations_Phase1B:acute-side-effects-questionnaire}}

\begin{savenotes}\sphinxattablestart
\centering
\begin{tabular}[t]{|*{2}{\X{1}{2}|}}
\hline
\sphinxstylethead{\sphinxstyletheadfamily 
\sphinxstyleemphasis{Parameter}
\unskip}\relax &\sphinxstylethead{\sphinxstyletheadfamily 
\sphinxstyleemphasis{Question}
\unskip}\relax \\
\hline
DateRecorded
&
The date when the information was originally recorded (the date of the subject’s visit).

This parameter is needed for the side-effects questionnaires which are initially recorded on paper and then subsequently transcribed electronically. For these questionnaires, the “Date” parameter corresponds to when the information was entered electronically, and the “DateRecorded” value corresponds to when the subject received the tES.
\\
\hline
Stimulation
&
Type of stimulation (tDCS, tRNS or Sham tDCS, Sham tRNS).

This question was removed from the questionnaire in late April 2015 because this subject information is already known through our subject condition assignment process (see the “Condition” parameter). It also required that the person administering the questionnaire be unblinded.
\\
\hline
Experimenter
&
Experimenter/Co-investigator who administered the questionnaire.
\\
\hline
PainPreSeverity
&
Are you experiencing any pain (headache, scalp pain, discomfort)?
\begin{itemize}
\item {} 
Question asked \sphinxstyleemphasis{before} stimulation (Absent, Mild, Moderate, Severe).

\end{itemize}
\\
\hline
PainPostSeverity
&\begin{itemize}
\item {} 
Question asked \sphinxstyleemphasis{after} stimulation (Absent, Mild, Moderate, Severe).

\end{itemize}
\\
\hline
PainPostRelationship
&\begin{itemize}
\item {} 
Relationship between subject’s pre and post pain severity, as assessed by senior staff (None, remote, Possible, Probable, Definite).

\end{itemize}
\\
\hline
PainComments
&
(text field)
\\
\hline&
\sphinxstyleemphasis{Repeat Pain Pre/Post/PostRelationship/Comments pattern with these two questions:}
\\
\hlineIrritationPreSeverity

IrritationPostSeverity

IrritationPostRelationship

IrritationComments
&
Is your scalp irritated (burning)?

{[}Experimenter assess scalp redness{]}
\\
\hlineConcentrationPreSeverity

ConcentrationPostSeverity

ConcentrationPostRelationship

ConcentrationComments
&
Are you having trouble concentrating?
\\
\hline
SensationsSeverity
&
Since the beginning of today’s session, have you felt sensations under the electrode locations (tingling, itching, burning, pain)?
\\
\hline
SensationsRelationship
&
Assessed by senior staff (None, remote, Possible, Probable, Definite).
\\
\hline
SensationsComments
&\\
\hline&
\sphinxstyleemphasis{Repeat Sensations Severity/Relationship/Comments pattern with these three questions:}
\\
\hlineNervousnessSeverity

NervousnessRelationship

NervousnessComments
&
Since the beginning of today’s session, have you felt nervous?
\\
\hlineNauseaSeverity

NauseaRelationship

NauseaComments
&
Since the beginning of today’s session, have you felt nauseous?
\\
\hlineOtherSeverity

OtherRelationship

OtherComments
&
Is there anything else that you would like to tell me?
\\
\hline
OtherEffect
&
Did the subject have any other adverse effect during or post-tES? (yes or No)
\\
\hline
OtherEffectComment
&
If YES then write a summary of the event below (500 char).
\\
\hline
\end{tabular}
\par
\sphinxattableend\end{savenotes}


\subsection{Multidimensional Mood State Questionnaire}
\label{\detokenize{Data_Definations_Phase1B:multidimensional-mood-state-questionnaire}}
All responses are answers to the statement “Right now I feel…” completed
by the word specified in the table below. Responses are selected from:
\begin{itemize}
\item {} 
Definitely not

\item {} 
Not

\item {} 
Not really

\item {} 
A little

\item {} 
Very much

\item {} 
Extremely

\end{itemize}


\begin{savenotes}\sphinxatlongtablestart\begin{longtable}{|l|l|}
\hline
\sphinxstylethead{\sphinxstyletheadfamily 
\sphinxstyleemphasis{Parameter}
\unskip}\relax &\sphinxstylethead{\sphinxstyletheadfamily 
\sphinxstyleemphasis{“Right now I feel…”}
\unskip}\relax \\
\hline
\endfirsthead

\multicolumn{2}{c}%
{\makebox[0pt]{\sphinxtablecontinued{\tablename\ \thetable{} -- continued from previous page}}}\\
\hline
\sphinxstylethead{\sphinxstyletheadfamily 
\sphinxstyleemphasis{Parameter}
\unskip}\relax &\sphinxstylethead{\sphinxstyletheadfamily 
\sphinxstyleemphasis{“Right now I feel…”}
\unskip}\relax \\
\hline
\endhead

\hline
\multicolumn{2}{r}{\makebox[0pt][r]{\sphinxtablecontinued{Continued on next page}}}\\
\endfoot

\endlastfoot

Content
&
Content
\\
\hline
Rested
&
Rested
\\
\hline
Restless
&
Restless
\\
\hline
Bad
&
Bad
\\
\hline
WornOut
&
Worn-out
\\
\hline
Composed
&
Composed
\\
\hline
Tired
&
Tired
\\
\hline
Great
&
Great
\\
\hline
Uneasy
&
Uneasy
\\
\hline
Energetic
&
Energetic
\\
\hline
Uncomfortable
&
Uncomfortable
\\
\hline
Relaxed
&
Relaxed
\\
\hline
Activated
&
Highly activated
\\
\hline
Superb
&
Superb
\\
\hline
Calm
&
Absolutely calm
\\
\hline
Sleepy
&
Sleepy
\\
\hline
Good
&
Good
\\
\hline
AtEase
&
At ease
\\
\hline
Unhappy
&
Unhappy
\\
\hline
Alert
&
Alert
\\
\hline
Discontent
&
Discontent
\\
\hline
Tense
&
Tense
\\
\hline
Fresh
&
Fresh
\\
\hline
Happy
&
Happy
\\
\hline
Nervous
&
Nervous
\\
\hline
Exhausted
&
Exhausted
\\
\hline
Calm
&
Calm
\\
\hline
Awake
&
Wide awake
\\
\hline
Wonderful
&
Wonderful
\\
\hline
Relaxed
&
Deeply relaxed
\\
\hline
\end{longtable}\sphinxatlongtableend\end{savenotes}


\subsection{Engagement Questionnaire}
\label{\detokenize{Data_Definations_Phase1B:engagement-questionnaire}}
Except where noted, subjects were asked to answer questions on a 1 (\sphinxstyleemphasis{Not
At All}) to 7 (\sphinxstyleemphasis{A Lot}) scale. In some questions, \sphinxstyleemphasis{Not At All} was
replaced by \sphinxstyleemphasis{Very Poor}. In some questions, \sphinxstyleemphasis{A Lot} was replaced by
\sphinxstyleemphasis{Very Much So,} \sphinxstyleemphasis{Very Aware, Very Difficult, Very Well or Definitely
Yes}.


\begin{savenotes}\sphinxatlongtablestart\begin{longtable}{|l|l|}
\hline
\sphinxstylethead{\sphinxstyletheadfamily 
\sphinxstyleemphasis{Parameter}
\unskip}\relax &\sphinxstylethead{\sphinxstyletheadfamily 
\sphinxstyleemphasis{Question}
\unskip}\relax \\
\hline
\endfirsthead

\multicolumn{2}{c}%
{\makebox[0pt]{\sphinxtablecontinued{\tablename\ \thetable{} -- continued from previous page}}}\\
\hline
\sphinxstylethead{\sphinxstyletheadfamily 
\sphinxstyleemphasis{Parameter}
\unskip}\relax &\sphinxstylethead{\sphinxstyletheadfamily 
\sphinxstyleemphasis{Question}
\unskip}\relax \\
\hline
\endhead

\hline
\multicolumn{2}{r}{\makebox[0pt][r]{\sphinxtablecontinued{Continued on next page}}}\\
\endfoot

\endlastfoot

Attention
&
To what extent did the game hold your attention?
\\
\hline
Focus
&
To what extent did you feel you were focused on the game?
\\
\hline
Effort
&
How much effort did you put into playing the game?
\\
\hline
Trying
&
Did you feel that you were trying your best?
\\
\hline
LoseTrackOfTime
&
To what extent did you lose track of time, e.g. did the game absorb your attention so that you were not bored?
\\
\hline
WorldAwareness
&
To what extent did you feel consciously aware of being in the real world whilst playing?
\\
\hline
EverydayConcerns
&
To what extent did you forget about your everyday concerns?
\\
\hline
Surroundings
&
To what extent were you aware of yourself in your surroundings?
\\
\hline
NoticeEvents
&
To what extent did you notice events taking place around you?
\\
\hline
UrgeToStop
&
Did you feel the urge at any point to stop playing and see what was happening around you?
\\
\hline
InteractingWithGame
&
To what extent did you feel that you were interacting with the game environment?
\\
\hline
SeparatedFromWorld
&
To what extent did you feel as though you were separated from your real-world environment?
\\
\hline
Fun
&
To what extent did you feel that the game was something fun you were experiencing, rather than a task you were just doing?
\\
\hline
GameStrongerThanWorld
&
To what extent was your sense of being in the game environment stronger than your sense of being in the real world?
\\
\hline
Involvement
&
At any point did you find yourself become so involved that you were unaware you were even using controls, e.g. it was effortless?
\\
\hline
OwnWill
&
To what extent did you feel as though you were moving through the game according to your own will?
\\
\hline
Challenging
&
To what extent did you find the game challenging?
\\
\hline
GiveUp
&
Were there any times during the game in which you just wanted to give up?
\\
\hline
Motivated
&
To what extent did you feel motivated while playing?
\\
\hline
Easy
&
To what extent did you find the game easy?
\\
\hline
MakingProgress
&
To what extent did you feel like you were making progress towards the end of the game?
\\
\hline
Performance
&
How well do you think you performed in the game?
\\
\hline
EmotionalAttachment
&
To what extent did you feel emotionally attached to the game?
\\
\hline
InterestGameProgress
&
To what extent were you interested in seeing how the game’s events would progress?
\\
\hline
WantToWin
&
How much did you want to “win” the game?
\\
\hline
Suspense
&
Were you in suspense about whether or not you would do well in the game?
\\
\hline
SpeakToGame
&
At any point did you find yourself become so involved that you wanted to speak to the game directly?
\\
\hline
EnjoyGraphics
&
To what extent did you enjoy the graphics and the imagery?
\\
\hline
EnjoyGame
&
How much would you say you enjoyed playing the game?
\\
\hline
DisappointedGameEnded
&
When it ended, were you disappointed that the game was over?
\\
\hline
PlayGameAgain
&
Would you like to play the game again?
\\
\hline
HowImmersed
&
How immersed did you feel? (1 \textendash{} Very immersed to 7 \textendash{} Not At All Immersed)
\\
\hline
ReadInstructions
&
Did you read all instructions completely before starting each task? (Yes or No)
\\
\hline
CommentReadInstructions
&
If No, please describe the reasons (250 char).
\\
\hline
UnderstandInstructions
&
Did you understand the task instructions? (1 \textendash{} Not at all to 7 \textendash{} Completely)
\\
\hline
CommentUnderstandInstructions
&
If there are tasks for which you did not completely understand the instructions, please specify how many there were and any further details you recall (250 char).
\\
\hline
EffortInstructions
&
How much mental effort did it take for you to follow the instructions? (1 \textendash{} Very Little to 7 \textendash{} A lot)
\\
\hline
\end{longtable}\sphinxatlongtableend\end{savenotes}


\subsection{Sleepiness Questionnaire \#1 (Pre)}
\label{\detokenize{Data_Definations_Phase1B:sleepiness-questionnaire-1-pre}}

\begin{savenotes}\sphinxattablestart
\centering
\begin{tabular}[t]{|*{2}{\X{1}{2}|}}
\hline
\sphinxstylethead{\sphinxstyletheadfamily 
\sphinxstyleemphasis{Parameter}
\unskip}\relax &\sphinxstylethead{\sphinxstyletheadfamily 
\sphinxstyleemphasis{Question}
\unskip}\relax \\
\hline
Sleepiness
&This is a quick way to assess how alert you are feeling. If it is during the day when you go about your business, ideally you would want a rating of a one. Take into account that most people have two peak times of alertness daily, at about 9 a.m. and 9 p.m. Alertness wanes to its lowest point at around 3 p.m.; after that it begins to build again. Rate your alertness at different times during the day. If you go below a three when you should be feeling alert, this is an indication that you have a serious sleep debt and you need more sleep.

Degree of sleepiness:
\begin{enumerate}
\item {} 
Feeling active, vital, alert, or wide awake

\item {} 
Functioning at high levels, but not at peak; able to concentrate

\item {} 
Awake, but relaxed; responsive but not fully alert

\item {} 
Somewhat foggy, let down

\item {} 
Foggy; losing interest in remaining awake; slowed down

\item {} 
Sleepy, woozy, fighting sleep; prefer to lie down

\item {} 
No longer fighting sleep, sleep onset soon; having dream-like thoughts

\item {} 
Asleep

\end{enumerate}
\\
\hline
Sleep
&
How many hours of sleep did you get last night?
\\
\hline
\end{tabular}
\par
\sphinxattableend\end{savenotes}


\subsection{Sleepiness Questionnaire \#2 (Post)}
\label{\detokenize{Data_Definations_Phase1B:sleepiness-questionnaire-2-post}}

\begin{savenotes}\sphinxattablestart
\centering
\begin{tabular}[t]{|*{2}{\X{1}{2}|}}
\hline
\sphinxstylethead{\sphinxstyletheadfamily 
\sphinxstyleemphasis{Parameter}
\unskip}\relax &\sphinxstylethead{\sphinxstyletheadfamily 
\sphinxstyleemphasis{Question}
\unskip}\relax \\
\hline
Sleepiness
&This is a quick way to assess how alert you are feeling. If it is during the day when you go about your business, ideally you would want a rating of a one. Take into account that most people have two peak times of alertness daily, at about 9 a.m. and 9 p.m. Alertness wanes to its lowest point at around 3 p.m.; after that it begins to build again. Rate your alertness at different times during the day. If you go below a three when you should be feeling alert, this is an indication that you have a serious sleep debt and you need more sleep.

Degree of sleepiness:
\begin{enumerate}
\item {} 
Feeling active, vital, alert, or wide awake

\item {} 
Functioning at high levels, but not at peak; able to concentrate

\item {} 
Awake, but relaxed; responsive but not fully alert

\item {} 
Somewhat foggy, let down

\item {} 
Foggy; losing interest in remaining awake; slowed down

\item {} 
Sleepy, woozy, fighting sleep; prefer to lie down

\item {} 
No longer fighting sleep, sleep onset soon; having dream-like thoughts

\item {} 
Asleep

\end{enumerate}
\\
\hline
\end{tabular}
\par
\sphinxattableend\end{savenotes}


\subsection{Alcohol \& Caffeine Questionnaire}
\label{\detokenize{Data_Definations_Phase1B:alcohol-caffeine-questionnaire}}

\begin{savenotes}\sphinxattablestart
\centering
\begin{tabular}[t]{|*{2}{\X{1}{2}|}}
\hline
\sphinxstylethead{\sphinxstyletheadfamily 
\sphinxstyleemphasis{Parameter}
\unskip}\relax &\sphinxstylethead{\sphinxstyletheadfamily 
\sphinxstyleemphasis{Question}
\unskip}\relax \\
\hline
Alcohol
&How many drinks containing alcohol have you consumed in the past 24 hours?

One standard drink is defined as:
\begin{enumerate}
\item {} 
12 ounces of beer (5\% alcohol content)

\item {} 
8 ounces of malt liquor (7\% alcohol content)

\item {} 
5 ounces of wine (12\% alcohol content)

\item {} 
1.5 ounces or a “shot” of 80-proof (40\% alcohol content) distilled spirits or liquor (e.g., gin, rum, vodka, whiskey)

\end{enumerate}
\\
\hline
Caffeine
&How many drinks containing caffeine have you consumed within an hour prior to your visit?

One standard drink is defined as:
\begin{enumerate}
\item {} 
6 ounces of tea

\item {} 
12 ounces of soda

\item {} 
8.5 ounces of Red Bull

\item {} 
3.5 ounces of coffee (Note: Starbucks “tall” coffee is 12 ounces)

\end{enumerate}
\\
\hline
\end{tabular}
\par
\sphinxattableend\end{savenotes}


\subsection{Handedness Questionnaire}
\label{\detokenize{Data_Definations_Phase1B:handedness-questionnaire}}
The subject was asked to respond to this instruction:
\begin{quote}

\sphinxstyleemphasis{Please indicate your preferences in the use of hands in the
following activities by selecting the appropriate button. Where the
preference is so strong that you would never try to use the other
hand unless absolutely forced to, select left/right hand “strongly
preferred.” If in any case you are really indifferent select “No
preference.”}

\sphinxstyleemphasis{Some of the activities require both hands. In these cases the part
of the task, or object, for which hand preference is wanted is
indicated in parentheses.}

\sphinxstyleemphasis{Please try to answer all the questions, and only leave a blank if
you have no experience at all of the object or task.}
\end{quote}

Responses to questions use this key:
\begin{quote}

1 \textendash{} Left hand strongly preferred

2 \textendash{} Left hand preferred

3 \textendash{} No preference

4 \textendash{} Right hand preferred

5 \textendash{} Right hand strongly preferred
\end{quote}


\begin{savenotes}\sphinxattablestart
\centering
\begin{tabulary}{\linewidth}[t]{|T|T|}
\hline
\sphinxstylethead{\sphinxstyletheadfamily 
\sphinxstyleemphasis{Parameter}
\unskip}\relax &\sphinxstylethead{\sphinxstyletheadfamily 
\sphinxstyleemphasis{Activity}
\unskip}\relax \\
\hline
Writing
&
Writing
\\
\hline
Drawing
&
Drawing
\\
\hline
Throwing
&
Throwing
\\
\hline
Scissors
&
Scissors
\\
\hline
Toothbrush
&
Toothbrush
\\
\hline
Knife
&
Knife (without fork)
\\
\hline
Spoon
&
Spoon
\\
\hline
Broom
&
Broom (upper hand)
\\
\hline
Match
&
Striking Match (match)
\\
\hline
OpenBox
&
Opening box (lid)
\\
\hline
\end{tabulary}
\par
\sphinxattableend\end{savenotes}


\subsection{Debriefing Questionnaire}
\label{\detokenize{Data_Definations_Phase1B:debriefing-questionnaire}}

\begin{savenotes}\sphinxattablestart
\centering
\begin{tabulary}{\linewidth}[t]{|T|T|}
\hline
\sphinxstylethead{\sphinxstyletheadfamily 
\sphinxstyleemphasis{Parameter}
\unskip}\relax &\sphinxstylethead{\sphinxstyletheadfamily 
\sphinxstyleemphasis{Question}
\unskip}\relax \\
\hline
ReceivingStimulation?
&
Do you think you were actually receiving electrical stimulation during training? (Yes or No)
\\
\hline
CommentStimulation
&
Comment (250 char).
\\
\hline
RotationStrategy?
&
While performing the Rotation Task that had you remember letters and arrows, did you employ a particular strategy? (Yes or No)
\\
\hline
CommentStrategy
&
If yes, please explain (500 char).
\\
\hline
\end{tabulary}
\par
\sphinxattableend\end{savenotes}


\section{Parameters That Are Specific to the EEG Files}
\label{\detokenize{Data_Definations_Phase1B:parameters-that-are-specific-to-the-eeg-files}}
There are three types of files with EEG content exported by the parser:
\begin{itemize}
\item {} 
EEG data files, they have a .easy suffix

\item {} 
Stimulation files, they have a .stim suffix

\item {} 
The parser generated eeg-info.csv file

\end{itemize}

The EEG data files and the stimulation files are generated by the
Neuroelectrics NIC application. The EEG data is recorded at 500 Hz and
the stimulation data is recorded at 1000 Hz. The stimulation file is
only generated for cases when a subject is being stimulated with tES,
including the sham conditions. The EEG data files and stimulation files
are posted as parser output although they are really created by the NIC.

For each EEG file, the NIC also generates an information file (.info
file suffix) which the parser uses for populating parameters in the
generated eeg-info file.

The NIC does not reliably generate .stim and .info files. We have
instances where one or the other file is missing.


\subsection{EEG Data Files}
\label{\detokenize{Data_Definations_Phase1B:eeg-data-files}}
The parser modifies the .easy file generated by the parser in these two
ways:
\begin{enumerate}
\item {} 
Normalizes and blinds the file name

\item {} 
Adds a column heading row to the EEG data file

\end{enumerate}

The .easy file name generated by the NIC should have this format:

\textless{}timestamp\textgreater{}\textless{}subject-id\textgreater{}\textless{}qualifier\textgreater{}.easy

The NIC generates the timestamp but the \textless{}subject-id\textgreater{} and \textless{}qualifier\textgreater{}
parts are entered by the experimenter and mistakes (including typos)
occur about 5\% of the time. The \textless{}qualifier\textgreater{} provides an indication of
what the subject was doing during the recording (pretest, training,
eyes-open, eyes-closed, etc.).

After correcting errors in the \textless{}subject-id\textgreater{} and \textless{}qualifier\textgreater{}, the parser
generates this normalized file name for the EEG data file:

\textless{}subject-id\textgreater{}\textless{}timestamp\textgreater{}\textless{}qualifier\textgreater{}.easy

Because EEG analysis started when many of the analysts were blinded, we
actually write the EEG data with blinded file names having this format:

\textless{}blinded-id\textgreater{}\textless{}seq-number\textgreater{}\textless{}qualifier\textgreater{}.easy

The \textless{}blinded-id\textgreater{} is a randomly selected integer that uniquely identifies
the subject, and the \textless{}seq-number\textgreater{} replaces the timestamp in a manner
that preserves numerical ordering. That is, if
\begin{quote}

\sphinxstyleemphasis{timestamp:sub:{}`1{}`} \textless{} \sphinxstyleemphasis{timestamp:sub:{}`2{}`}
\end{quote}

Then the corresponding sequence numbers will also follow this relation:
\begin{quote}

\sphinxstyleemphasis{seqnumber:sub:{}`1{}`} \textless{} \sphinxstyleemphasis{seqnumber:sub:{}`2{}`}
\end{quote}

The EEG-sum file shows all three names for each EEG file, the original
name provided by the NIC, the normalized file name (which we don’t use),
and the blinded file name which we use for naming the exported EEG data
files.

When there is a stimulation file corresponding to an EEG data file, we
write that file with the same name as the EEG data file but with a .stim
file extension:

\textless{}blinded-id\textgreater{}\textless{}seq-number\textgreater{}\textless{}qualifier\textgreater{}.stim

These are the file qualifiers used for naming the blinded files:


\begin{savenotes}\sphinxattablestart
\centering
\begin{tabulary}{\linewidth}[t]{|T|T|}
\hline
\sphinxstylethead{\sphinxstyletheadfamily 
\sphinxstyleemphasis{Qualifier}
\unskip}\relax &\sphinxstylethead{\sphinxstyletheadfamily 
\sphinxstyleemphasis{Description}
\unskip}\relax \\
\hline
pretest-eo
&
Prior to the subject’s pretest, the 5 minute eyes-open recording.
\\
\hline
pretest-ec
&
Prior to the subject’s pretest, the 5 minute eyes-closed recording.
\\
\hline
pretest-test
&
The subject’s pretest.
\\
\hline
training\sphinxstyleemphasis{i}-pretec
&
Prior to the subject’s \sphinxstyleemphasis{i}$^{\text{th}}$ training, the 5 minute eyes-closed recording. Expected on training sessions 3 and 8 for subjects trained by Honeywell, Northeastern and Oxford.
\\
\hline
training\sphinxstyleemphasis{i}-train
&
EEG recorded during the subject’s \sphinxstyleemphasis{i}$^{\text{th}}$ training. If the subject received tES, this file will contain just the portion of the training when tES was being applied. Otherwise it will contain EEG for the entire training session.
\\
\hline
training\sphinxstyleemphasis{i}-posteeg
&
EEG recorded during the subject’s \sphinxstyleemphasis{i}$^{\text{th}}$ training, after the completion of tES. This file will not exist if the subject did not receive tES during the session.
\\
\hline
training\sphinxstyleemphasis{i}-postec
&
After the subject’s \sphinxstyleemphasis{i}$^{\text{th}}$ training, the 5 minute eyes-closed recording. Expected on training sessions 3 and 8 for subjects trained by Honeywell, Northeastern and Oxford.
\\
\hline
training\sphinxstyleemphasis{i}-train
&
Prior to the subject’s pretest, the 5 minute eyes-open recording.
\\
\hline
\end{tabulary}
\par
\sphinxattableend\end{savenotes}

The only change we make to the contents of the EEG data file is to add a
row with column headings for the EEG data.

We make no changes to the contents of the stimulation files.

These are the columns in the EEG data files, ordered from first to last:


\begin{savenotes}\sphinxattablestart
\centering
\begin{tabulary}{\linewidth}[t]{|T|T|}
\hline
\sphinxstylethead{\sphinxstyletheadfamily 
\sphinxstyleemphasis{Parameter}
\unskip}\relax &\sphinxstylethead{\sphinxstyletheadfamily 
\sphinxstyleemphasis{Description}
\unskip}\relax \\
\hline
\textless{}channel\textgreater{}
&
The first eight columns (StarStim), or first 20 columns (Enobio 20), or 32 columns (Enobio 32) contain EEG data. If the parser found a .info file corresponding to the EEG data file, these column headers are populated from the montage described in the .info file. Otherwise, the generic names \sphinxstyleemphasis{Chan1, Chan2, …} are used.
\\
\hlineAccel1

Accel2

Accel3
&
Accelerometer sensor values.
\\
\hline
Trigger
&
A trigger value injected by the pretest, posttest or training application being used by the subject.
\\
\hline
NeTime
&
A timestamp associated for the EEG data values. The value is provided by a StarStim or Enobio headset, and expresses milliseconds since the start of the UNIX-defined epoch.
\\
\hline
\end{tabulary}
\par
\sphinxattableend\end{savenotes}


\subsection{The Generated eeg-sum File}
\label{\detokenize{Data_Definations_Phase1B:the-generated-eeg-sum-file}}
The following table describes the contents of the generated eeg-sum
file. Unless otherwise noted, values are extracted from the information
file (.info) expected for each EEG data file. Values are left blank when
this file is missing.

The parser can generate a blinded version of the eeg-sum file which
omits these four columns:
\begin{itemize}
\item {} 
FileDate

\item {} 
Notes

\item {} 
OriginalFile

\item {} 
NormalizedFile

\end{itemize}


\begin{savenotes}\sphinxatlongtablestart\begin{longtable}{|*{2}{\X{1}{2}|}}
\hline
\sphinxstylethead{\sphinxstyletheadfamily 
\sphinxstyleemphasis{Parameter}
\unskip}\relax &\sphinxstylethead{\sphinxstyletheadfamily 
\sphinxstyleemphasis{Description}
\unskip}\relax \\
\hline
\endfirsthead

\multicolumn{2}{c}%
{\makebox[0pt]{\sphinxtablecontinued{\tablename\ \thetable{} -- continued from previous page}}}\\
\hline
\sphinxstylethead{\sphinxstyletheadfamily 
\sphinxstyleemphasis{Parameter}
\unskip}\relax &\sphinxstylethead{\sphinxstyletheadfamily 
\sphinxstyleemphasis{Description}
\unskip}\relax \\
\hline
\endhead

\hline
\multicolumn{2}{r}{\makebox[0pt][r]{\sphinxtablecontinued{Continued on next page}}}\\
\endfoot

\endlastfoot

AccelChans
&
Number of accelerometer channels, either zero or three. (From EEG information file.)
\\
\hline
AddChan
&
Presence of an “Additional Channel”. (From EEG information file.)
\\
\hline
BlindedFile
&
The blinded name for the generated EEG data file.
\\
\hline
Device
&
The device type, either \sphinxstyleemphasis{StarStim}, \sphinxstyleemphasis{StarStim (EEG only mode)}, \sphinxstyleemphasis{Enobio20}, or \sphinxstyleemphasis{Enobio32}. (From EEG information file.)
\\
\hline
Duration
&
Duration of the EEG file, expressed in hh:mm:ss format.
\\
\hline
EegChans
&
Number of EEG recording channels. (From EEG information file.)
\\
\hline
EegRecs
&
Number of EEG records in the file. (From EEG information file.)
\\
\hline
EogCorr
&
Status of the EOG correction filter. (From EEG information file.)
\\
\hline
FileDate
&
Date when EEG data was recorded expressed in local time.
\\
\hline
FW
&
The version of the firmware in the NECBOX.
\\
\hline
InfoFile
&
Set to TRUE when the parser was able to find the information (.info) file corresponding to the EEG data (.easy).
\\
\hline
LineFilter
&
Status of the line filter.
\\
\hline
LostSamples
&
Number of missing EEG data records from the file.
\\
\hline
MAC
&
The NECBOX’s MAC address (uniquely identifies the specific unit).
\\
\hline
Montage
&
The EEG montage used for connecting EEG sensors to the headcap. (From EEG information file.)
\\
\hline
NIC
&
The NIC software version. (From EEG information file.)
\\
\hline
NormalizedFile
&
The normalized name for the generated EEG data file (not used since we name the generated EEG data files with the BlindedFile name).
\\
\hline
Notes
&
Manually entered notes about the EEG data file.
\\
\hline
OriginalFile
&
The file name of the original EEG data file.
\\
\hline
PacketsLost
&
The number of packets (and percent of total) sent via Bluetooth by the NECBOX that were not received by the NIC. (From EEG information file.)
\\
\hline
Qualifier
&
The second part of the EEG file qualifier (the first part can be inferred from the “Period” parameter.
\\
\hline
Rating
&A subjective evaluation that rates the file against three metrics. The metrics are:
\begin{itemize}
\item {} 
Duration \textendash{} the EEG recording is longer “D+” or shorter “D-“ than expected

\item {} 
Triggers \textendash{} the file contains more “T+” or fewer “T-“ triggers than expected

\item {} 
Trigger rate \textendash{} relative to the file’s duration, the file contains more “R+” or fewer “R-” triggers than expected.

\end{itemize}

A blank rating means that the file’s duration and triggers conform to expectations.
\\
\hline
ShamRampDown
&
The sham ramp down time, expressed in seconds. (From EEG information file.)
\\
\hline
ShamScore
&The likelihood that the subject was receiving sham stimulation. A score of less than -1 suggests that the subject is very likely to have received sham stimulation while a score greater than 1 suggests that the subject is very unlikely to have received sham stimulation. Values in the range -1 .. 1 cannot be used to infer the sham condition. File duration may accurately determine sham condition however.

A blank value is used for files that show no evidence of stimulation, sham or otherwise.

Note: this column may not be generally useful since the result is factored into the StimError column: incorrect sham conditions are reported as sham\_yes and sham\_no, and cases where the sham condition could not be determined are reported as sham\_unverified.
\\
\hline
StimChans
&
The number of StarStim channels used for stimulation. (From EEG information file.)
\\
\hline
StimDuration
&
The duration of the stimulation protocol, expressed in seconds. (From EEG information file.)
\\
\hline
StimError
&
Deviations from the expected tES for this file type and for this subject. No stimulation is expected for all file types except training\sphinxstyleemphasis{i}-train when \sphinxstyleemphasis{i} is greater than 2 (training without tES). Values are composed from these clauses:
\begin{itemize}
\item {} 
no\_stim: tES was expected but not detected

\item {} 
tDCS: tDCS stimulation was detected when tRNS or no stimulation was expected

\item {} 
tRNS: tRNS stimulation was detected when tDCS or no stimulation was expected

\item {} 
sham\_no: sham tES was expected but subject received either tDCS or tRNS stimulation

\item {} 
sham\_yes: tDCS or tRNS stimulation was expected but subject received sham stimulation instead

\item {} 
sham\_unverified: we are unable to determine with confidence if subject received tES or sham tES.

\end{itemize}
\\
\hline
StimFile
&
Set to TRUE when the parser found a stimulation file (.stim suffix) corresponding to the EEG data file.
\\
\hline
StimProtocol
&
The name of the stimulation template used for controlling stimulation. (From EEG information file.)
\\
\hline
StimRampDown
&
The stimulation ramp down time, expressed in seconds. (From EEG information file.)
\\
\hline
StimRampUp
&
The stimulation ramp up time, expressed in seconds. (From EEG information file.)
\\
\hline
StimRecs
&
The number of stimulation records. (From EEG information file.)
\\
\hline
StimType
&
The type of stimulation (tDCS or tRNS). (From EEG information file.)
\\
\hline
Triggers
&
The number of triggers (non-zero trigger values) found in the EEG data file.
\\
\hline
\end{longtable}\sphinxatlongtableend\end{savenotes}


\chapter{Parser File “rf-triggers”}
\label{\detokenize{Data_Definations_Phase1B:parser-file-rf-triggers}}
The generated “rf-triggers” file provides linkage for a RobotFactory
trigger value found in an EEG data file back to the game play
information in a parser generated “robotfactory” file. The “rf-triggers”
file is intended to be used with a specific Matlab utility and is not
likely to be useful outside the context of that utility.

For a training session, RobotFactory creates an “output\_log” file and
an “LSL\_Testing” file while also sending trigger values for the NIC to
merge into the EEG stream. The “output\_log” file contains much of the
information provided in the parser generated “robotfactory” files. The
“LSL\_Testing” file contains the trigger values sent to the NIC along
with timestamps and other information relating to the trigger. The
generated “rf-triggers” file contains much of the information in the
“LSL\_Testing” file.

The TrialID column was added to the generated “robotfactory” file to
support this capability. It simply numbers the lines in the file.


\section{Contents of the “rf-triggers” File}
\label{\detokenize{Data_Definations_Phase1B:contents-of-the-rf-triggers-file}}
The parser builds a line in the “rf-triggers” file for each
“LSL\_Testing” file. Consequently a line spans an entire training
session (and so can be very long \textendash{} remember that this file is intended
to be used by another tool).

The line begins with an approximation for the blinded EEG file name for
the training session. The name is approximate because typically there is
more than one EEG file and because, given the timestamp in the
RobtFactory output\_log file, the parser can at best infer an
approximate time for the EEG data file. To emphasize that the file name
is approximate, it is prefixed by ‘\$’.

So the intent is to have another (and yet to be implemented)
semi-automated utility that, given the approximate file name could scan
the generated “eeg-sum” file and generate a list of likely EEG data
files for the training session. It seems prudent for someone to check
the generated list.

Following the approximate EEG file name, the line contains a 4-tuple for
each trigger sent to the NIC. The 4-tuple contains these values:
\begin{itemize}
\item {} 
A label that describes the purpose of the trigger (e.g.,
begin\_shift, begin\_trial, begin\_stimulus,
stimulus\_begins\_exiting)

\item {} 
The trigger value, a 32 bit value where the high order bit is always
set (to distinguish from triggers not generated by RobotFactory which
never have the high order bit set)

\item {} 
A trial id that identifies the row in the “robotfactory” collection
of files that corresponds to the trial that contains this trigger, or
a negative number which denotes the following cases:
\begin{itemize}
\item {} 
-1: the trigger is outside the bounds of a trial (before first
trial, after last trial, between trials)

\item {} 
-2: there is a corresponding output\_log file, but it does not
contain any shifts

\item {} 
-3: the output\_log file does not contain this shift

\item {} 
-4: the output\_log does not contain this trial

\end{itemize}

\item {} 
A timestamp for the trigger

\end{itemize}


\chapter{Publications}
\label{\detokenize{About::doc}}\label{\detokenize{About:publications}}

\chapter{Indices and tables}
\label{\detokenize{index:indices-and-tables}}\begin{itemize}
\item {} 
\DUrole{xref,std,std-ref}{genindex}

\item {} 
\DUrole{xref,std,std-ref}{modindex}

\item {} 
\DUrole{xref,std,std-ref}{search}

\end{itemize}



\renewcommand{\indexname}{Index}
\printindex
\end{document}